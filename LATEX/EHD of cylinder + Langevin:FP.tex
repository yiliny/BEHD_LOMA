%!TEX program = pdflatex
\documentclass[books,12pt]{elegantpaper}
\usepackage{import}
\usepackage{physics}
\usepackage{amsthm,amsmath,amssymb}
\usepackage{mathrsfs}
%\usepackage{graphics}
\usepackage{cancel} %用于在偏微分符号上画斜线
\usepackage{ulem} %波浪线, 双下划线
\usepackage{tikz} % 数字画圈
\newcommand*\circled[1]{\tikz[baseline=(char.base)]{
            \node[shape=circle,draw,inner sep=1.5pt] (char) {#1};}}


\usepackage{xcolor}
\usepackage{sectsty}
\definecolor{ChimieBlue}{rgb}{0.282,0.514,0.6}
\sectionfont{\color{ChimieBlue}}
\subsectionfont{\color{ChimieBlue}}
%\subsubsectionfont{\color{ChimieBlue}}


\usepackage{newcommand_yye}



\title{\textcolor{ChimieBlue}{Brownian motion near a Soft Surface}}
\author{\href{https://yiliny.github.io/yiliny/}{Yilin YE}$^{\rm a,b}$, Yacine AMAROUCHENE$^{\rm a}$, David DEAN$^{\rm a}$, and Thomas SALEZ$^{\rm a}$ \\ \small{\rm{a.} Univ. Bordeaux, CNRS, LOMA, UMR 5798, F-33405, Talence, France} \\ \small{\rm{b.} Département de Chimie, École Normale Supérieure, Univ. PSL, F-75230, Paris, France}}
\date{\today} %将时间括号中内容留白,这样编译之后文档中不显示时间。
%\underline{Yilin YE}$^{\rm a,b*}$, Yacine AMAROUCHENE$^{\rm a}$, David DEAN$^{\rm a}$ and Thomas SALEZ$^{\rm a}$

\begin{document}

\maketitle

%\begin{abstract}
%What would happen if a fluid-immersed negatively buoyant particle moves randomly in 3D, especially in the vicinity of a thin compressible elastic wall in 2D? Herein, I do my M1S2 internship at LOMA (Laboratoire Ondes et Matière d'Aquitaine) with Prof. Thomas SALEZ, focusing on this mixing Langevin + ElastoHydroDynamic Problems.

%In this personal note, I try to derive important equations from the relevant articles published previously, adding fundamental and useful concepts at the same time. Then, paramount details would be discussed precisely, as well as the new improvement over the old system.
%\end{abstract}
\tableofcontents %本句作用为显示目录
%\thispagestyle{empty} %本句作用为去掉首页/目录页的页码




\newpage
\setcounter{page}{1} %本句作用为将该页页码设置为1



\section{Framework Introduction}

%% Below the subsection "Theoretical Fundamentals"
\import{}{BEHD_1-1.tex}


%% Below the subsection "Salez2015: Elastohydrodynamics of a sliding, spinning and sedimenting cylinder near a soft wall"
\import{}{BEHD_1-2.tex}


%% Below the subsection "David's note: Determining noise from deterministic forces"
\import{}{BEHD_1-3.tex}





\newpage
\sct{Brownian Motion in 3D ($\Delta, X, \Theta$) near a Soft Surface}
\label{BEHD}

Now consider the 3D Brownian motion (perpendicular, vertical, rotation) near a soft surface based on the results shown in the subsection \ref{ssc:Thomas}, with the figure below:
\begin{figure}[htbp]\centering
\includegraphics[width=0.5\textwidth]{plots/newfromGM1404_3.jpg}
\caption{Brownian motion near a soft surface}
\end{figure}

We exploit Langevin equation to describe the Brownian motion, with the following expression:
$$ m\dot{v} = - \xi v + \delta F $$
where the first term refers to systematic put of the environment influence, and the second term $\delta F$ refers to fluctuation / random put. As the Gaussian white noise, there is no correlation in space and in time so $\llang \delta F(t) \rrang = 0$, and $\llang \delta F(t_1) \delta F(t_2) \rrang = 2B \delta(t_1 - t_2)$. Suppose $\gamma = \xi /m$, we have
$$ \dot{v} = -\gamma v + \frac{\delta F}{m} $$
With Laplace transform, 
$$ \tilde{v}(s) = \frac{v(0)}{s+\gamma} + \frac{\delta \tilde{F}(s)}{m(s+\gamma)} $$
and then the inverse transform
$$ v(t) = v(0) e^{-\gamma t} + \int_0^t dt' \frac{\delta F(t')}{m} \exp\left[-\gamma(t-t')\right] $$

At long time limit, $\langle v^2 (t) \rangle \to \frac{k_B T}{m}$. According to the expression above, we could calculate 
$$ \begin{align}
\langle v^2(t) \rangle &= \langle v^2(0) \rangle e^{-2\gamma t} + 2 e^{-\gamma t} \int_0^t dt' \frac{\langle \delta F(t) v(0) \rangle}{m} e^{-\gamma (t-t')} + \int_0^t dt_1 \int_0^t dt_2 e^{-\gamma (t-t_1)} e^{-\gamma (t-t_2)} \frac{\langle \delta F(t_1) \delta F(t_2) \rangle}{m^2} \\ % rang 1
&= \langle v^2(0) \rangle e^{-2\gamma t} + \int_0^t dt_1 e^{-2\gamma (t-t_1)} \frac{2B}{m^2} = \langle v^2(0) \rangle e^{-2\gamma t} - \frac{1}{2\gamma} (e^{-2\gamma t} - 1) \times \frac{2B}{m^2} \\ % rang 2
&= \langle v^2(0) \rangle e^{-2\gamma t} + \frac{B}{\zeta m} (1 - e^{-2\gamma t}) % rang 3
\end{align} $$
thus leading to $ B = k_B T \zeta $ or $\langle \delta F(0) \delta F(t) \rangle = 2k_B T \zeta \delta(t)$. We then write $\delta F(t) = \sqrt{2 k_B T \gamma m} \eta(t)$, following that $\llang \eta(0) \eta(t) \rrang = \delta(t)$. 
$$ m\dot{v} = -m \gamma v + \sqrt{\frac{2\gamma m}{\beta}} \eta(t) $$
Since $\eta(t)$ is not differentiable, we could define integrals of the process, introducing $W(t)$ as a Wiener process such that $\eta(t) = \mrm{d}W/\mrm{d}t$.
Consider the external force $f(t)$, we finally writes
$$ dv = \frac{f(t)}{m} dt - \gamma v dt + \sqrt{\frac{2\gamma}{\beta m}} dW(t) $$
%where $f(t)$ contains the external forces, $\gamma$ would be a matrix rather than a constant for $v$ is a velocity vector, the last term shows the random force.

For our case, 3D Brownian motion should be taken into account. Therefore, additional formula would be needed, including the velocity $\vec{v}$ on each direction, then the mass $\vec{m}$, and $\gamma$ as a matrix.
In this section, we would first consider the second derivatives in Thomas' results, writing the friction matrix; then consider the effective "mass" on different directions, showing $\frac{1}{m}$ as an inverse matrix; finally the amplitude for the noise correlator. %one numerical procedure to simulate the Langevin equation.


%% Below the subsection "New coefficients \lambda & \Gamma"
\import{}{BEHD_2-1.tex}


%% Below the subsection "New coefficients \gamma"
%\import{}{BEHD_2-2.tex}
% Discard in 20220420, combine that into other files.


%% Below the subsection "Mass vector"
\import{}{BEHD_2-2.tex}


%% Below the subsection "Noise correlator amplitude"
\import{}{BEHD_2-3.tex}


%% Below the subsection "MSD"
\import{}{BEHD_2-4.tex}




\newpage
\sct{Numerical Practice}

%% Below the subsection "Discretisation algorithm"
\import{}{BEHD_3-1.tex}
%\ssc{Update $\gamma_\rm{eff}}






%% Below the subsection "Possible modification"
\import{}{BEHD_3-3.tex}




%%%%% 计划 %%%%%
% 2-1 单位质量摩擦矩阵;
% 2-2 考虑质量矩阵后摩擦矩阵;(即加入转动惯量)
% 2-3 噪音振幅,考虑零阶然后代入微扰
% 2-4 对角化?

% 3-0 简单讨论?加入现在3.3内容
% 3-1 数值结果作图?
% 3-2 时间关联函数//来不及
% 3-3 概率分布函数//来不及





\nocite{EINAV2010,Havrylchyk2018} 

\bibliographystyle{aer}
\bibliography{wp_ref}
\end{document}

