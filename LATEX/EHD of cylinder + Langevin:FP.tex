%!TEX program = pdflatex
\documentclass[books,12pt]{elegantpaper}
\usepackage{physics}
\usepackage{amsthm,amsmath,amssymb}
\usepackage{mathrsfs}
%\usepackage{graphics}
\usepackage{cancel} %用于在偏微分符号上画斜线
\usepackage{ulem} %波浪线, 双下划线
\usepackage{tikz} % 数字画圈
\newcommand*\circled[1]{\tikz[baseline=(char.base)]{
            \node[shape=circle,draw,inner sep=1.5pt] (char) {#1};}}


\usepackage{xcolor}
\usepackage{sectsty}
\definecolor{ChimieBlue}{rgb}{0.282,0.514,0.6}
\sectionfont{\color{ChimieBlue}}
\subsectionfont{\color{ChimieBlue}}
%\subsubsectionfont{\color{ChimieBlue}}


\usepackage{newcommand_yye}



\title{\textcolor{ChimieBlue}{Mixing Langevin + ElastoHydroDynamic of cylinder}}
%\author{\href{}{Yilin YE}}
\author{\href{https://yiliny.github.io/yiliny/}{{Yilin YE}}\\ \textit{\small{Département de Chimie, École Normale Supérieure, 75231 Paris Cedex 05, France}} \\ \textit{\small{Laboratoire Ondes et Matière d'Aquitaine, 33405 Talence Cedex, France}}}
\date{\today} %将时间括号中内容留白,这样编译之后文档中不显示时间。

\begin{document}

\maketitle

%\begin{abstract}
%What would happen if a fluid-immersed negatively buoyant particle moves randomly in 3D, especially in the vicinity of a thin compressible elastic wall in 2D? Herein, I do my M1S2 internship at LOMA (Laboratoire Ondes et Matière d'Aquitaine) with Prof. Thomas SALEZ, focusing on this mixing Langevin + ElastoHydroDynamic Problems.

%In this personal note, I try to derive important equations from the relevant articles published previously, adding fundamental and useful concepts at the same time. Then, paramount details would be discussed precisely, as well as the new improvement over the old system.
%\end{abstract}
\tableofcontents %本句作用为显示目录
%\thispagestyle{empty} %本句作用为去掉首页/目录页的页码




\newpage
\setcounter{page}{1} %本句作用为将该页页码设置为1



\section{Framework Introduction}
\ssc{Theoretical Fundamentals}

\tit{Here are fundamentals used in the following parts, including Gibbs-Boltzmann distribution, Langevin equation, Fokker-Planck equation.}

\parag{Gibbs-Boltzmann distribution} The Boltzmann distribution is a probability distribution that gives the probability of a certain state as a function of that state's energy and temperature of the system to which the distribution is applied. It is given as
$$ p_i = \frac{\exp(-\beta \veps_i)}{\sum_{j=1}^M \exp(-\beta \veps_j)} $$

\parag{Langevin equation}  %In physics, a Langevin equation (named after Paul Langevin) is a stochastic differential equation describing how a system evolves when subjected to a combination of deterministic and fluctuating ("random") forces. The dependent variables in a Langevin equation typically are collective (macroscopic) variables changing only slowly in comparison to the other (microscopic) variables of the system. The fast (microscopic) variables are responsible for the stochastic nature of the Langevin equation. One application is to Brownian motion, which models the fluctuating motion of a small particle in a fluid.

The original Langevin equation describes Brownian motion, the apparently random movement of a particle in a fluid due to collisions with molecules of the fliud, 
$$ m \frac{dv}{dt} = - \lambda v + \eta(t) $$
where $v$ is the velocity of the particle, and $m$ is the mass. The force acting on the particle is written as a sum of a viscous force proportional to the particles's velocity, and a noise term $\eta(t)$ representing the effect of the collisions with the molecules of the fluid. The force $\eta(t)$ has a Gaussian probability distribution with correlation function $ \llang \eta_i(t) \eta_j(t^\prime) \rrang = 2 \lambda k_B T \delta_{ij} \delta(t-t^\prime)$

There are two common choices of discretization: the Itô and the Stratonovich conventions. Discretization of the Langevin equation:
$$ \frac{x_{t+\Delta} - x_t}{\Delta} = -V^\prime(x_t) + \xi_t $$
with an associated discretization of the correlations:
$$ \llang f\left[x(t)\right] \rrang \to \llang f(x_t) \rrang \fives \llang f\left[x(t)\right] \xi(t) \rrang \to \llang f(x_t)\xi_t \rrang \fives \llang f\left[x(t)\right] \dot{x}(t) \rrang \to \llang f(x_t) \frac{x_{t+\Delta} - x_t}{\Delta} \rrang $$
which leads to \textbf{Itô's chain rule}:
$$ \frac{d}{dt} \llang f\left[x(t)\right] \rrang = \llang f^\prime\left[x(t)\right] \frac{dx}{dt} \rrang + T \llang f^{\prime\prime} \left[ x(t) \right] \rrang $$


\parag{Fokker-Planck equation} In one spatial dimension $x$, for an Itô process driven by the standard Wiener process $W_t$ and described by the stochastic differential equation (SDE)
$$ dX_t = \mu(X_t,t) dt + \sigma(X_t,t) dW_t $$
with drift $\mu(X_t,t)$ and diffusion coefficient $D(X_t,t) = \sigma^2(X_t,t)/2$, the Fokker-Planck equation for the probability density $p(x,t)$ of the random variable $X_t$ is
$$ \pder{t} p(x,t) = - \pder{x} \left[ \mu(x,t) p(x,t) \right] + \pdv[2]{}{x} \left[ D(x,t) p(x,t) \right] $$
\uu{\tit{Derivation from the over-damped Langevin equation}}\\
Let $\bb{P}(x,t)$ be the probability density density function to find a particle in $\left[x, x + dx\right]$ at time $t$, and let $x$ satisfy:
$$ \dot{x}(t) = -V^\prime(x) + \xi (t) $$
if $f$ is a function, we have:
$$ \frac{d}{dt} \llang f\left[ x(t) \right] \rrang = \frac{d}{dt} \int \bb{P}(x,t) f(x) dx = \int \pder[\bb{P}(x,t)]{t} f(x) dx $$
but using Itô's chain rule:
$$ \frac{d}{dt} \llang f\left[x(t)\right] \rrang = \llang f^\prime \left[ x(t) \right] \frac{dx}{dt} \rrang + T \llang f^{\prime\prime} \left[ x(t) \right] \rrang $$
with Langevin's equation
$$ \frac{d}{dt} \llang f\left[x(t)\right] \rrang = \llang f^\prime \left[ x(t) \right] \left\{ - V^\prime \left[ x(t) \right] + \xi(t) \right\} \rrang + T \llang f^{\prime\prime} \left[ x(t) \right] \rrang $$
since $\llang f^\prime \left[ x(t) \right] \xi(t) \rrang = 0$, we have
$$ \frac{d}{dt} \llang f \left[ x(t) \right] \rrang = \int \left[ \frac{df(x)}{dx} \left( - \frac{dV(x)}{dx} \right) + T \frac{d^2 f(x)}{dx^2} \right] \bb{P}(x,t) dx $$
performing an integration by parts, and using that $\bb{P}(x,t)$ is a probability density vanishing at $x\to\infty$:
$$ \int \pder[\bb{P}(x,t)]{t} f(x) dx = \int \pder{x} \left[ \frac{dV(x)}{dx} + T \pder{x} \right] \bb{P}(x,t) f(x) dx $$
this is true for any function $f$, thus
$$ \boxed{ \pder[\bb{P}(x,t)]{t} = \pder{x} \left[ \frac{dV(x)}{dx} + T \pder{x} \right] \bb{P}(x,t) } $$
It could be written as $\partial_t \bb{P}(x,t) = - H_{FP} \bb{P}(x,t)$ with $H_{FP}$ the Fokker-Planck operator shown above.
%We try to solve $\partial_t \bb{P}(x,t) = 0$. A good guess is the Gibbs-Boltzmann probability density, 





\ssc{Salez2015: Elastohydrodynamics of a sliding, spinning and sedimenting cylinder near a soft wall}
\label{ssc:Thomas}

\tit{Here we look at Thomas' publication (arxiv: 1412.0162) on Journal of Fluid Mechanics, 779 181 (2015). This article describes the sedimentation, sliding, and spinning motions of a cylinder near a thin compressible elastic wall by thin-film lubrication dynamics. Below is the illustration: } 
% We consider the motion of a fluid-immersed negatively buoyant particle in the vicinity of a thin compressible elastic wall, a situation that arises in a variety of technological and natural settings. We use scaling arguments to establish different regimes of sliding, and complement these estimates us- ing thin-film lubrication dynamics to determine an asymptotic theory for the sedimentation, sliding, and spinning motions of a cylinder. The resulting theory takes the form of three coupled nonlinear singular-differential equations. Numerical integration of the resulting equations confirms our scaling relations and further yields a range of unexpected behaviours. 

\begin{center}
\includegraphics[scale=1.0,frame]{plots/Salez2015figure.pdf}
\end{center}

The deformation of the soft wall reads:
\beq \delta_s(x,t) = - \frac{h_s p(x,t)}{2\mu + \lambda} \tlag{Salez2015.2} \eeq
% ... we non-dimensionalize the problem using the following choices: 
as well as the dimensionless parameters:
$z=Zr\veps$, $h=Hr\veps$, $\delta = \Delta r\veps$, $x = Xr\sqrt{2\veps}$, $x_G=X_G r \sqrt{2\veps}$, $\theta = \Theta\sqrt{2\veps}$, $t=Tr\sqrt{2\veps}/c$, $u=Uc$, and $p=P\eta c\sqrt{2}/(r\veps^{3/2})$; a free fall velocity scale $c=\sqrt{2gr\rho^\ast / \rho}$ and one dimensionless parameter $\xi$ measures the ratio of the free fall time $\sqrt{\rho r \epsilon / (\rho^\ast g)}$ and the typical lubrication damping time $m\epsilon^{3/2}/\eta$ over which the inertia of the cylinder vanishes.
\beq \xi = \frac{3\sqrt{2} \eta}{r^{3/2} \veps \sqrt{\rho \rho^\ast g}} \tlag{Salez2015.3}
\beq \kappa = \frac{2h_s \eta \sqrt{g\rho^\ast}}{r^{3/2} \veps^{5/2} (2\mu+\lambda)\sqrt{\rho}} \tlag{Salez2015.5}

With perturbation theory in first-order correction of $\kappa$, the soft compressible wall gives 
\beq \ddot{X}_G + \frac{2\varepsilon \xi}{3} \frac{\dot{X}_G}{\sqrt\Delta} + \frac{\kappa \varepsilon \xi}{6} \left[ \frac{19}{4} \frac{\dot\Delta \dot{X}_G}{\Delta^{7/2}} - \frac{\dot\Delta \dot\Theta}{\Delta^{7/2}} + \frac{1}{2} \frac{\ddot\Theta - \ddot{X}_G}{\Delta^{5/2}} \right] - \sqrt{\frac{\varepsilon}{2}} \sin\alpha = 0 \tlag{Salez2015.50} \eeq
\beq \ddot{\Delta} + \xi \frac{\dot{\Delta}}{\Delta^{3/2}} + \frac{\kappa\xi}{4} \left[ 21 \frac{\dot{\Delta}^2}{\Delta^{9/2}} - \frac{(\dot\Theta - \dot{X}_G)^2}{\Delta^{7/2}} - \frac{15}{2} \frac{\ddot\Delta}{\Delta^{7/2}} \right] + \cos\alpha = 0  \tlag{Salez2015.51} \eeq
\beq \ddot{\Theta} + \frac{4\veps\xi}{3} \frac{\dot\Theta}{\sqrt\Delta} + \frac{\kappa \veps \xi}{3} \left[ \frac{19}{4} \frac{\dot\Delta \dot\Theta}{\Delta^{7/2}} - \frac{\dot\Delta \dot{X}_G}{\Delta^{7/2}} + \frac{1}{2} \frac{\ddot{X}_G - \ddot\Theta}{\Delta^{5/2}} \right] = 0 \tlag{Salez2015.52} \eeq
where $\Delta$ refers to $z$ and $X_G$ refers to $x$ after the scaling. For the plan case, we set $\alpha=0$. %Also, there would be no rotation, thus $\theta(t)=0$.
%$$ \ddot{X}_G + \frac{2\varepsilon \xi}{3} \frac{\dot{X}_G}{\sqrt\Delta} + \frac{\kappa \varepsilon \xi}{6} \left[ \frac{19}{4} \frac{\dot\Delta \dot{X}_G}{\Delta^{7/2}} + \frac{1}{2} \cdot \frac{ - \ddot{X}_G}{\Delta^{5/2}} \right] = 0 $$
%$$ \ddot{\Delta} + \xi \frac{\dot{\Delta}}{\Delta^{3/2}} + \frac{\kappa\xi}{4} \left[ 21 \frac{\dot{\Delta}^2}{\Delta^{9/2}} - \frac{( - \dot{X}_G)^2}{\Delta^{7/2}} - \frac{15}{2} \frac{\ddot\Delta}{\Delta^{7/2}} \right] + \cos\alpha = 0  $$



\ssc{David's note: Determining noise from deterministic forces}
\textit{Here is the note of David Dean, considering the Brownian motion but only in two dimension ($\Delta, X$). The rotation had been neglected ($\dot\Theta=0$), and the second derivatives in the first-order correction of $\kappa$ as well. To be clear, Fokker-Planck equation would be carefully discussed. Other personal comments are also written in Italic.}

Consider the following deterministic equations \tit{($\alpha$ refers to $\Delta, X$ these two directions)}
\beq dX_\alpha = V_\alpha dt \tag{David.1} \label{David.1} \eeq
and \tit{($\mathbf{X},\mathbf{V}$ refer to the position and the velocity, respectively)}
\beq dV_\alpha = -U_\alpha dt - \nabla \phi(\mathbf{X}) dt \tlag{David.2} \eeq 
We assume that $U_\alpha$ are generated by hydrodynamic interactions which do not however affect the equilibrium Gibbs-Boltzmann distribution which is \beq P_{eq} (\mathbf{X},\mathbf{V}) = \frac{1}{\bar{Z}} \exp \left( - \frac{\beta \mathbf{V}^2}{2} - \beta \phi(\mathbf{X}) \right) \tlag{David.3} \eeq

\textit{Exploit the Fokker-Planck operator ($\cdots$ refers the similar terms about $X_\alpha$)}
$$ \pder[P]{t} = - H_{FP} P = \pder{x} \left[ \frac{dV}{dx} P + T \pder{x} P \right] = \pder{V_\alpha} \left[ (U_\alpha + \nabla_\alpha \phi) P + T \gamma_{\alpha\beta} \pder[P]{V_\beta} \right] + \pder{X_\alpha} \left[ \cdots \right] $$
\tit{Note $\pder[P]{X_\alpha} = P \left( -\beta \pder[\phi]{X_\alpha} \right)$ and $\pder[P]{V_\alpha} = P \left( -\beta V_\alpha \right)$. Consider the gravity $\phi(\mathbf{X}) = - mg\Delta$, and then we could derive the eq. \ref{David.4}, regarding $k_B$ as 1}
$$ \bgn
\pder{X_\alpha} \left[ \frac{dV}{dx} P + T \pder{x} P \right] &= \pder{X_\alpha} \left[ \frac{dV}{dX_\alpha} P + T \pder{X_\alpha} P + T \pder{V_\alpha} P \right] \\
&= \pder{X_\alpha} \left[ (\nabla_\alpha\phi)P + \cancel{T} \cdot P \left( -\cancel{\beta} \pder[\phi]{X_\alpha} \right) + T \pder{V_\alpha} P \right] = \pder{X_\alpha} \left[ T \pder{V_\alpha} P \right] \\
&= \pder{X_\alpha} \left[ \cancel{T} \cdot P \left( -\cancel\beta V_\alpha \right) \right] = - \pder{X_\alpha} V_\alpha P
\end{align} $$

The Fokker Planck equation at finite temperature which introduces white noise and possibly temperature dependent drifts is $\phi(\mathbf{X})$ is 
\beq \pder[P]{t} = \pder{V_\alpha} \left[ T \gamma_{\alpha\beta} \pder[P]{V_\beta} + U_\alpha P + \pder[\phi]{X_\alpha} P \right] - \pder{X_\alpha} V_\alpha P  \tlag{David.4} \eeq
\tit{The last two terms would vanish since}
$$ \pder{V_\alpha} \left( \pder[\phi]{X_\alpha} P \right) = \cancel{ \left( \pder{V_\alpha} \pder[\phi]{X_\alpha} \right) } \cdot P + \pder[\phi]{X_\alpha} \cdot \pder[P]{V_\alpha} = \pder[\phi]{X_\alpha} \cdot P (-\beta V_\alpha) $$
$$ \pder{X_\alpha} V_\alpha P = \cancel{ \left( \pder[V_\alpha]{X_\alpha} \right) } P + V_\alpha \left( \pder[P]{X_\alpha} \right) = V_\alpha \cdot P \cdot \left( -\beta \pder[\phi]{X_\alpha} \right) $$
\tit{Therefore, at equilibrium $\pder[P]{t} = 0$}
$$ \pder[P]{t} = \pder{V_\alpha} \left[ T \gamma_{\alpha\beta} \pder[P]{V_\beta} + U_\alpha P \right] = \pder{V_\alpha} \left[ \cancel{T} \gamma_{\alpha\beta} P \cdot (- \cancel\beta V_\beta) + U_\alpha P \right] = \pder{V_\alpha} \left[ (U_\alpha - \gamma_{\alpha\beta} V_\beta) \cdot P \right] = 0 $$

We obtain the GB distribution for the steady state if 
\beq U_\alpha = \gamma_{\alpha\beta} V_\beta \tlag{David.5} \eeq
We have for small velocities that
\beq U_\alpha = \lambda_{\alpha\beta} (\mathbf{X}) V_\beta + \Lambda_{\alpha\beta\gamma} (\mathbf{X}) V_\beta V_\gamma \tlag{David.6} \eeq
and so we find
\beq \gamma_{\alpha\beta} V_\beta = \lambda_{\alpha\beta}(\mathbf{X}) V_\beta + \Lambda_{\alpha\beta\gamma} (\mathbf{X}) V_\beta V_\gamma \tlag{David.7} \eeq
Written this way the term $\lambda_{\alpha\beta}(\mathbf{X})$ is just the friction tensor in the absence of any elastic effects. We can thus write
\beq \gamma_{\alpha\beta} = \lambda_{\alpha\beta} + \gamma_{2\alpha\beta} \tlag{David.8} \eeq
and we write
\beq \gamma_{2\alpha\beta} = \Gamma_{\alpha\beta\gamma} V_\gamma \tlag{David.9} \eeq
and
\beq \Gamma_{\alpha\beta\gamma} (\mathbf{X}) V_\beta V_\gamma = \Lambda_{\alpha\beta\gamma} (\mathbf{X}) V_\beta V_\gamma \tlag{David.10} \eeq
where we without loss of generality take $\Lambda_{\alpha\beta\gamma} = \Lambda_{\alpha\gamma\beta}$, which then gives
\beq \Gamma_{\alpha\beta\gamma} + \Gamma_{\alpha\gamma\beta} = 2 \Lambda_{\alpha\beta\gamma} \tlag{David.11} \eeq
We have to solve this system with the constraint that $\Gamma_{\alpha\beta\gamma} V_\gamma = \Gamma_{\beta\alpha\gamma} V_\gamma$. In Thomas' problem \tit{(see subsection \ref{ssc:Thomas})} we have
$$ \ddot{\Delta} + \xi \frac{\dot{\Delta}}{\Delta^{3/2}} + \frac{\kappa\xi}{4} \left[ 21 \frac{\dot{\Delta}^2}{\Delta^{9/2}} - \frac{(\dot\Theta - \dot{X}_G)^2}{\Delta^{7/2}} - \frac{15}{2} \frac{\ddot\Delta}{\Delta^{7/2}} \right] + \cos\alpha = 0 $$
$$ \ddot{X}_G + \frac{2\varepsilon \xi}{3} \frac{\dot{X}_G}{\sqrt\Delta} + \frac{\kappa \varepsilon \xi}{6} \left[ \frac{19}{4} \frac{\dot\Delta \dot{X}_G}{\Delta^{7/2}} - \frac{\dot\Delta \dot\Theta}{\Delta^{7/2}} + \frac{1}{2} \frac{\ddot\Theta - \ddot{X}_G}{\Delta^{5/2}} \right] - \sqrt{\frac{\varepsilon}{2}} \sin\alpha = 0 $$
\tit{where $\Delta$ refers to $z$ and $X_G$ refers to $x$. Note $\dot\Delta=-U_z$ and $\dot{X}_G=-U_x$, we write}
\beq U_z = \xi \frac{V_z}{Z^{3/2}} + \frac{21 \kappa \xi}{4} \frac{V_z^2}{Z^{9/2}} - \frac{\kappa \xi}{4} \frac{V_x^2}{Z^{7/2}}  \tlag{David.12} \eeq
\beq U_x = 2 \xi\veps \frac{V_x}{3Z^{1/2}} + \frac{19 \kappa \xi \veps}{24} \frac{V_z V_x}{Z^{7/2}} \tlag{David.13} \eeq
\textit{Note $\dot\Delta = V_z$ and $\dot{X}_G = V_x$, we could extract the relevant coefficients. But Attention! Here $\dot\Theta$ was assumed 0, and the second derivatives in first-order correlation had been ignored.} \\
Form this we find that
\beq \bgn
\sum_{\alpha\beta} \Lambda_{z\alpha\beta} V_\alpha V_\beta &= \frac{21 \kappa \xi}{4} \frac{V_z^2}{Z^{9/2}} - \frac{\kappa \xi}{4} \frac{V_x^2}{Z^{7/2}} \\
\sum_{\alpha\beta} \Lambda_{x\alpha\beta} V_\alpha V_\beta &= \frac{19 \kappa \xi \veps}{24} \frac{V_z V_x}{Z^{7/2}}
\end{align} \tlag{David.14} \eeq
This gives the set of equations
\beq \Gamma_{zzz} = \frac{21 \kappa \xi}{4 Z^{9/2}} \tlag{David.15} \eeq
\beq \Gamma_{zxx} = - \frac{\kappa \xi}{4Z^{7/2}} \tlag{David.16} \eeq

\beq \Gamma_{zxz} + \Gamma_{zzx} = 0 \tlag{David.17} \eeq
\beq \Gamma_{xzz} = 0 \tlag{David.18} \eeq
\beq \Gamma_{xxx} = 0 \tlag{David.19} \eeq
\beq \Gamma_{xxz} + \Gamma_{xzx} = \frac{19 \kappa \xi \veps}{24 Z^{7/2}} \tlag{David.20} \eeq
The symmetry $\Gamma_{\alpha\beta\gamma} = \Gamma_{\beta\alpha\gamma}$ now gives
\beq \Gamma_{xxz} = \frac{19 \kappa \xi \veps}{24 Z^{7/2}} - \Gamma_{xzx} = \frac{19 \kappa \xi \veps}{24 Z^{7/2}} - \Gamma_{zxx} = \frac{\kappa\xi}{Z^{7/2}} \left( \frac{19\veps}{24} + \frac{1}{4} \right) \tlag{David.21} \eeq
as well as
\beq \Gamma_{zxz} = \Gamma_{zzx} = 0 \tlag{David.22} \eeq
The Langevin equation corresponding to this is, using the Itô convention,
\beq \frac{dV_\alpha}{dt} = -U_\alpha - \pder[\phi(\mathbf{X})]{X_\alpha} + T \pder[\gamma_{\alpha\beta}]{V_\beta} + \eta_\alpha(t) \tlag{David.23} \eeq
which can be written as
\beq \frac{dV_\alpha}{dt} = -U_\alpha - \pder[\phi(\mathbf{X})]{X_\alpha} + T \Gamma_{\alpha\beta\beta} + \eta_\alpha(t) \tlag{David.24} \eeq
where we use the Einstein summation convention and the noise correlator is given by
\beq \llang \eta_\alpha (t) \eta_\beta(t^\prime) \rrang = 2T \gamma_{\alpha \beta} \delta(t-t^\prime) = 2T \left[ \lambda_{\alpha\beta}(\mathbf{X}) + \Gamma_{\alpha\beta\gamma}(\mathbf{X}) V_\gamma \right] \delta(t-t^\prime) \tlag{David.25} \eeq
Putting this together we find \tit{(from eq. \ref{David.24}) with all $\Gamma_{\alpha\beta\beta}$ only depending on $\Delta, X$.}
\beq \bgn
\frac{dV_z}{dt} &= -V^\prime(Z) - \xi \frac{V_z}{Z^{3/2}} - \frac{21\kappa\xi}{4}\frac{V_z^2}{Z^{9/2}} + \frac{\kappa \xi V_x^2}{4 Z^{7/2}} + T \left[ \frac{21 \kappa \xi}{4 Z^{9/2}} - \frac{\kappa \xi}{4 Z^{7/2}} \right] + \eta_z(t) \\
\frac{dV_x}{dt} &= -2\xi\veps \frac{V_x}{3 Z^{1/2}} - \frac{19 \kappa \xi \evps V_z V_x}{24 Z^{7/2}} + \eta_x(t) 
\end{align} \tlag{David.26} \eeq


\sct{Further Analyses}
Generally, the Brownian motion would furnish the following equation:
$$ dv = \frac{f(t)}{m} dt - \gamma v dt + \sqrt{\frac{2\gamma}{\beta m}} dW(t) $$
where $f(t)$ contains the external forces, $\gamma$ would be a matrix rather than a constant for $v$ is a velocity vector, the last term shows the random force.

In this section, we would first consider the second derivatives in Thomas' results, updating the $\gamma$ matrix; then consider the effective "mass" on different directions ($\Delta, X, \Theta$), showing $\frac{1}{m}$ as an inverse matrix; finally one numerical procedure to simulate the Langevin equation.

\ssc{New $\lambda_{\alpha\beta}, \Gamma_{\alpha\beta\gamma}$ in 3D ($\Delta, X, \Theta$)}
In this part, we would renew coefficients for the motion in 3D ($\Delta, X, \Theta$). Based on the previous subsection, we could repeat the calculation by Fokker-Planck operator, finding the similar results with additional terms about $\Theta$.

For the sake of convenience, we re-write Thomas'  differential equations (see subsection \ref{ssc:Thomas}) with $\dot{v}_i$, (and $X$ refers to $X_G$)
\beq \bgn
-U_Z = \dot{v}_\Delta = \ddot\Delta &= F_\Delta (\Delta,v_\Delta,v_X,v_\Theta,\dot{v}_\Delta) \\ % rang 1
-U_X = \dot{v}_X = \ddot{X} &= F_X (\Delta,v_\Delta,v_X,v_\Theta,\dot{v}_X,\dot{v}_\Theta)  \\ % rang 2
-U_\Theta = \dot{v}_\Theta = \ddot\Theta &= F_\Theta (\Delta,v_\Delta,v_X,v_\Theta,\dot{v}_X,\dot{v}_\Theta)  % rang 3
\end{align} \tlag{Yilin.1} \eeq
However, we'd like to derive equations for each $\dot{v}$ only depending on $\Delta$ and $v$, without $\dot{v}$. %namely 
%In 3D system, we have independent coordinates $\Delta,v_\Delta,v_{X},v_\Theta$. 
%\beq \dot{v}_i = F_i + TF^{drift}_i + \eta_i \eeq
%where $F_i$ would originate from external potential. 
Therefore, we have to find the proper expression for each $\dot{v}_i$. 

Consider the second derivative in the eq. \ref{Salez2015.51}, 
\beq \ddot{\Delta} + a_1 \frac{\dot{\Delta}}{\Delta^{3/2}} + a_2 \frac{\dot{\Delta}^2}{\Delta^{9/2}} + a_3 \frac{\dot\Theta^2}{\Delta^{7/2}} + a_3 \frac{\dot{X}^2}{\Delta^{7/2}} + a_4 \frac{\dot\Theta \dot{X}}{\Delta^{7/2}} + a_5 \frac{\ddot\Delta}{\Delta^{7/2}} + a_6 = 0 \tlag{Yilin.2} \eeq
\beq \ddot\Delta = \left( a_1 \frac{\dot{\Delta}}{\Delta^{3/2}} + a_2 \frac{\dot{\Delta}^2}{\Delta^{9/2}} + a_3 \frac{\dot\Theta^2}{\Delta^{7/2}} + a_3 \frac{\dot{X}^2}{\Delta^{7/2}} + a_4 \frac{\dot\Theta \dot{X}}{\Delta^{7/2}} + a_6 \right) \times \frac{-1}{1 + a_5 / \Delta^{7/2}} \tlag{Yilin.3} \eeq
We know that $a_1 = \xi$, $a_2 = \frac{21\kappa\xi}{4}$, $a_3 = -\frac{\kappa\xi}{4}$, $a_4 = \frac{\kappa\xi}{2}$, $a_5 = -\frac{15\kappa\xi}{8}$, $a_6 = \cos(\alpha=0) = 1$. After simple calculation, we could obtain $\dot{v}_\Delta$ ($\dot{v}_z$) namely $\ddot\Delta$


%Fokker-Planck eq.
%$$ T \sum_{\alpha\beta} \pder{Y_\alpha} \left[ \gamma_{\alpha\beta} \pder{Y_\beta} P \right] + \pder{Y_\alpha} \left[ -F_\alpha P \right] = \pder[P]{t} = 0 $$
%the matrix $\gamma_{\alpha\beta}$ is symmetric and thus there are 10 independent elements.
%$$ \dot{Y}_\alpha = \left[ F_\alpha + T \pder{Y_\beta} \gamma_{\beta\alpha} \right] + \sqrt{2T} \eta_\alpha $$
%where the white noise $\eta_\alpha$ satisfies
%$$ \llang \eta_\alpha(t) \eta_\beta(t^\prime) \rrang = \delta(t-t^\prime) \gamma_{\alpha\beta}(t) $$


%\beq \ddot{X}_G + \frac{2\varepsilon \xi}{3} \frac{\dot{X}_G}{\sqrt\Delta} + \frac{\kappa \varepsilon \xi}{6} \left[ \frac{19}{4} \frac{\dot\Delta \dot{X}_G}{\Delta^{7/2}} - \frac{\dot\Delta \dot\Theta}{\Delta^{7/2}} + \frac{1}{2} \frac{\ddot\Theta - \ddot{X}_G}{\Delta^{5/2}} \right] - \sqrt{\frac{\varepsilon}{2}} \sin\alpha = 0 \tlag{Salez2015.50} \eeq


\beq -\dot{v}_\Delta = U_z = \noindent\(\frac{8 \Delta ^{9/2}+2 \xi  \left(-\Delta  \kappa  v_X^2+4 \Delta ^3 v_z+21 \kappa  v_z^2+2 \Delta  \kappa  v_X v_{\theta }-\Delta  \kappa v_{\theta }^2\right)}{8 \Delta ^{9/2}-15 \Delta  \kappa  \xi }\) \tlag{Yilin.4} \eeq %\end{doublespace}

Similarly, we write the eqs \ref{Salez2015.50} and \ref{Salez2015.52} as
\beq \textcolor{red}{\ddot{X}} + b_1 \frac{\dot{X}}{\sqrt\Delta} + b_2  \frac{\dot\Delta \dot{X}}{\Delta^{7/2}} + b_3 \frac{\dot\Delta \dot\Theta}{\Delta^{7/2}} + b_4 \frac{\textcolor{blue}{\ddot{\Theta}}}{\Delta^{5/2}} + b_5 \frac{\textcolor{red}{\ddot{X}}}{\Delta^{5/2}} + b_6 = 0 \tlag{Yilin.5} \eeq
\beq \textcolor{blue}{\ddot{\Theta}} + c_1 \frac{\dot\Theta}{\sqrt\Delta} + c_2 \frac{\dot\Delta \dot\Theta}{\Delta^{7/2}} + c_3 \frac{\dot\Delta \dot{X}}{\Delta^{7/2}} + c_4 \frac{\textcolor{red}{\ddot{X}}}{\Delta^{5/2}} + c_5 \frac{\textcolor{blue}{\ddot{\Theta}}}{\Delta^{5/2}} + c_6 = 0 \tlag{Yilin.6} \eeq
with all coefficients we need: $b_1 = \frac{2\veps\xi}{3}$, $b_2 = \frac{19\kappa\xi\veps}{24}$, $b_3 = - \frac{\kappa\xi\veps}{6}$, $b_4 = \frac{\kappa\xi\veps}{12}$, $b_5 = - \frac{\kappa\xi\veps}{12}$, $b_6 = \sin(\alpha = 0) = 0$; and $c_1 = \frac{4\veps\xi}{3}$, $c_2 = \frac{19\kappa\xi\veps}{12}$, $c_3 = - \frac{\kappa\xi\veps}{3}$, $c_4 = \frac{\kappa\xi\veps}{6}$, $c_5 = - \frac{\kappa\xi\veps}{6}$, $c_6 = 0$. For this system of linear equations, the coefficient matrix has full rank.
$$ \left( \begin{array}{cc} 1 + (b_5) & (b_4) \\(c_4) & 1+(c_5)\end{array} \right) \left(\begin{array}{cc}\ddot{X} \\ \ddot\Theta  \end{array}\right) = \left(\begin{array}{cc} (b_1+b_2+b_3+b_6) \\ (c_1+c_2+c_3+c_6) \end{array}\right) $$
Then we could solve $\ddot{X}=\dot{v}_X$ and $\ddot\Theta = \dot{v}_\Theta$ directly %mark
%\beq \ddot{\Delta} + \xi \frac{\dot{\Delta}}{\Delta^{3/2}} + \frac{\kappa\xi}{4} \left[ 21 \frac{\dot{\Delta}^2}{\Delta^{9/2}} - \frac{(\dot\Theta - \dot{X}_G)^2}{\Delta^{7/2}} - \frac{15}{2} \frac{\ddot\Delta}{\Delta^{7/2}} \right] + \cos\alpha = 0  \tlag{Salez2015.51} \eeq
%\beq \ddot{\Theta} + \frac{4\veps\xi}{3} \frac{\dot\Theta}{\sqrt\Delta} + \frac{\kappa \veps \xi}{3} \left[ \frac{19}{4} \frac{\dot\Delta \dot\Theta}{\Delta^{7/2}} - \frac{\dot\Delta \dot{X}_G}{\Delta^{7/2}} + \frac{1}{2} \frac{\ddot{X}_G - \ddot\Theta}{\Delta^{5/2}} \right] = 0 \tlag{Salez2015.52} \eeq
\beq \bgn
&- \dot{v}_X = U_X = \\
& \noindent\(\frac{\epsilon  \xi  \left(\kappa  \left(16 \Delta ^3 \epsilon  \xi +\left(-24 \Delta ^{5/2}+23 \epsilon  \kappa  \xi \right) v_z\right)
v_{\theta }+v_X \left(-4 \epsilon  \kappa ^2 \xi  v_z+\left(6 \Delta ^{5/2}-\epsilon  \kappa  \xi \right) \left(16 \Delta ^3+19 \kappa  v_{\theta
}\right)\right)\right)}{36 \left(4 \Delta ^6-\Delta ^{7/2} \epsilon  \kappa  \xi \right)}\)
\end{align} \tlag{Yilin.7} \eeq
\beq \bgn
& - \dot{v}_\Theta = U_\Theta = \\
& \noindent\(\frac{\epsilon  \xi  \left(\left(16 \Delta ^3 \left(12 \Delta ^{5/2}-\epsilon  \kappa  \xi \right)+\kappa  \left(228 \Delta ^{5/2}-23
\epsilon  \kappa  \xi \right) v_z\right) v_{\theta }+\kappa  v_X \left(\left(-48 \Delta ^{5/2}+4 \epsilon  \kappa  \xi \right) v_z+\epsilon  \xi
 \left(16 \Delta ^3+19 \kappa  v_{\theta }\right)\right)\right)}{36 \left(4 \Delta ^6-\Delta ^{7/2} \epsilon  \kappa  \xi \right)}\)
\end{align} \tlag{Yilin.8} \eeq

Compare with the eq. \ref{Yilin.1}, we finally remove the second derivatives inside each expression
\beq \bgn
\dot{v}_\Delta &= F_\Delta (\Delta,v_\Delta,v_X,v_\Theta)  \\ % rang 1
\dot{v}_X &= F_X (\Delta,v_\Delta,v_X,v_\Theta) \\ % rang 2
\dot{v}_\Theta &= F_\Theta (\Delta,v_\Delta,v_X,v_\Theta) % rang 3
\end{align} \tlag{Yilin.9} \eeq
See eqs \ref{David.5} $\sim$ \ref{David.9}, we could extract these $\lambda_{\alpha\beta}$ by
\beq \lambda_{\alpha\beta} = \rm{Coefficient}[ U_\alpha , v_\beta ] - \rm{Coefficient}[ U_\alpha , v_\beta v_\gamma ] \times v_\gamma \tlag{Yilin.10} \eeq
and $\Gamma_{\alpha\beta\beta}$ by
\beq \Gamma_{\alpha\beta\beta} = \rm{Coefficient}[ U_\alpha , v_\beta v_\beta ] \tlag{Yilin.11} \eeq
As for $\Gamma_{\alpha\beta\gamma}$, we should resolve them by
\beq 2\Lambda_{\alpha\beta\gamma} = \rm{Coefficient}[ U_\alpha , v_\beta v_\gamma ] = \Gamma_{\alpha\beta\gamma} + \Gamma_{\alpha\gamma\beta} \tlag{Yilin.12} \eeq
as well as the constraint $\Gamma_{\alpha\beta\gamma} = \Gamma_{\beta\alpha\gamma}$.

After some calculations verified by \textit{Mathematica}, we list all $\lambda_{\alpha\beta}$ 

\beq \begin{align}
\lambda_{\text{zz}} &= \noindent\(\frac{8 \Delta ^2 \xi }{8 \Delta ^{7/2}-15 \kappa  \xi }\) \\
\lambda_{\text{xx}} &= \noindent\(-\frac{4 \epsilon  \xi  \left(-6 \Delta ^{5/2}+\epsilon  \kappa  \xi \right)}{36 \Delta ^3-9 \sqrt{\Delta } \epsilon  \kappa  \xi }\) \\
\lambda_{\theta \theta} &= \noindent\(-\frac{4 \epsilon  \xi  \left(-12 \Delta ^{5/2}+\epsilon  \kappa  \xi \right)}{36 \Delta ^3-9 \sqrt{\Delta } \epsilon  \kappa  \xi }\)
\end{align} \tlag{Yilin.13} \eeq

\beq \lambda_{x\theta} = \lambda_{\theta x} = \noindent\(\frac{4 \epsilon ^2 \kappa  \xi ^2}{36 \Delta ^3-9 \sqrt{\Delta } \epsilon  \kappa  \xi }\) \tlag{Yilin.14} \eeq
\beq \lambda_{zx} = \lambda_{xz} = \lambda_{z\theta} = \lambda_{\theta z} = 0 \tlag{Yilin.15} \eeq

and then $\Gamma_{\alpha\beta\gamma}$
\beq \bgn
\Gamma _{\text{zzz}} &= \noindent\(\frac{42 \kappa  \xi }{8 \Delta ^{9/2}-15 \Delta  \kappa  \xi }\) \\
\Gamma _{\text{xzx}} = \Gamma _{\text{zxx}} &= \noindent\(\frac{2 \kappa  \xi }{-8 \Delta ^{7/2}+15 \kappa  \xi }\) \\
\Gamma _{\text{$\theta z \theta $}} = \Gamma _{\text{z$\theta \theta $}} &= \noindent\(\frac{2 \kappa  \xi }{-8 \Delta ^{7/2}+15 \kappa  \xi }\)
\end{align} \tlag{Yilin.16} \eeq
\beq \Gamma_{zxz} = \Gamma_{zzx} = \Gamma_{zz\theta} = \Gamma_{z\theta z} = 0 \tlag{Yilin.17} \eeq
\beq \Gamma_{xzz} = \Gamma_{xxx} = \Gamma_{\theta zz} = \Gamma_{\theta\theta\theta} = 0 \tlag{Yilin.18} \eeq
\beq \Gamma_{\theta xx} = \Gamma_{x \theta x} = \Gamma_{x \theta\theta} = \Gamma_{\theta x \theta} = 0 \tlag{Yilin.19} \eeq

\beq \bgn
\Gamma _{\text{xxz}}&= \frac{1}{9} \kappa  \xi  \left(\frac{18}{8 \Delta ^{7/2}-15 \kappa  \xi }+\frac{\epsilon ^2 \kappa  \xi }{-4 \Delta ^6+\Delta ^{7/2} \epsilon \kappa  \xi }\right) \\
\Gamma _{\text{xx$\theta $}}&= \frac{19 \epsilon  \kappa  \xi  \left(-6 \Delta ^{5/2}+\epsilon  \kappa  \xi \right)}{-144 \Delta ^6+36 \Delta ^{7/2} \epsilon  \kappa \xi } \\
\Gamma _{\text{$\theta \theta $x}}&= \frac{19 \epsilon ^2 \kappa ^2 \xi ^2}{36 \left(4 \Delta ^6-\Delta ^{7/2} \epsilon  \kappa  \xi \right)} \\
\Gamma _{\text{$\theta \theta $z}}&= \frac{\epsilon  \kappa  \xi  \left(-228 \Delta ^{5/2}+23 \epsilon  \kappa  \xi \right)}{-144 \Delta ^6+36 \Delta ^{7/2} \epsilon  \kappa \xi }
\end{align} \tlag{Yilin.20} \eeq

\beq \bgn
\Gamma _{\text{zx$\theta $}}=\Gamma _{\text{xz$\theta $}}=& -\frac{25}{18 \Delta }-\frac{19 \epsilon  \kappa  \xi }{72 \Delta ^{7/2}}+\frac{2 \kappa  \xi }{8 \Delta ^{7/2}-15 \kappa  \xi }+\frac{50 \Delta ^{3/2}}{36 \Delta ^{5/2}-9 \epsilon  \kappa  \xi } \\
\Gamma _{\text{z$\theta $x}}=\Gamma _{\text{$\theta $zx}}=& \frac{25}{18 \Delta }+\frac{19 \epsilon  \kappa  \xi }{72 \Delta ^{7/2}}+\frac{2 \kappa  \xi }{8 \Delta ^{7/2}-15 \kappa  \xi }+\frac{50 \Delta ^{3/2}}{9 \left(-4 \Delta ^{5/2}+\epsilon  \kappa  \xi \right)} \\
\Gamma _{\text{x$\theta $z}}=\Gamma _{\text{$\theta $xz}}=& \frac{2}{15}-\frac{1}{2 \Delta }-\frac{3 \epsilon  \kappa  \xi }{8 \Delta ^{7/2}}+\frac{16}{15 \left(-8+\frac{15 \kappa  \xi }{\Delta^{7/2}}\right)}+\frac{1}{2 \Delta -\frac{\epsilon  \kappa  \xi }{2 \Delta ^{3/2}}}
\end{align} \tlag{Yilin.21} \eeq



\ssc{$\gamma_{\alpha\beta}$ and linear approximation of $\kappa$} \label{gamma}
Since $\gamma_{\alpha\beta} = \lambda_{\alpha\beta} + \Gamma_{\alpha\beta\gamma} V_\gamma$, we have 


\beq \bgn
\gamma_{zz} &= \noindent\(\frac{8 \Delta ^2 \xi }{8 \Delta ^{7/2}-15 \kappa  \xi }+\frac{42 \kappa  \xi  v_z}{8 \Delta ^{9/2}-15 \Delta  \kappa  \xi }\) \\
&= \noindent\(\frac{\xi }{\Delta ^{3/2}}+\left(\frac{15 \xi ^2}{8 \Delta ^5}+\frac{21 \xi  v_z}{4 \Delta ^{9/2}}\right) \kappa +\left(\frac{225 \xi
^3}{64 \Delta ^{17/2}}+\frac{315 \xi ^2 v_z}{32 \Delta ^8}\right) \kappa ^2+O[\kappa ]^3\)
\end{align} \tlag{Yilin.22} \eeq


\beq \bgn
\gamma_{zx} = \gamma_{xz} &= \noindent\(\frac{2 \kappa  \xi  v_X}{-8 \Delta ^{7/2}+15 \kappa  \xi }+\left(-\frac{25}{18 \Delta }-\frac{19 \epsilon  \kappa  \xi }{72 \Delta ^{7/2}}+\frac{2
\kappa  \xi }{8 \Delta ^{7/2}-15 \kappa  \xi }+\frac{50 \Delta ^{3/2}}{36 \Delta ^{5/2}-9 \epsilon  \kappa  \xi }\right) v_{\theta }\) \\
&= \noindent\(-\frac{\left(\xi  \left(3 v_X-3 v_{\theta }-\epsilon  v_{\theta }\right)\right) \kappa }{12 \Delta ^{7/2}}+\frac{5 \xi ^2 \left(-27 v_X+27
v_{\theta }+5 \Delta  \epsilon ^2 v_{\theta }\right) \kappa ^2}{288 \Delta ^7}+O[\kappa ]^3\)
\end{align} \tlag{Yilin.23} \eeq


\beq \bgn
\gamma_{z\theta} = \gamma_{\theta z} &=\noindent\(\left(\frac{25}{18 \Delta }+\frac{19 \epsilon  \kappa  \xi }{72 \Delta ^{7/2}}+\frac{2 \kappa  \xi }{8 \Delta ^{7/2}-15 \kappa  \xi }+\frac{50
\Delta ^{3/2}}{9 \left(-4 \Delta ^{5/2}+\epsilon  \kappa  \xi \right)}\right) v_X+\frac{2 \kappa  \xi  v_{\theta }}{-8 \Delta ^{7/2}+15 \kappa  \xi
}\) \\
&= \noindent\(-\frac{\left(\xi  \left(-3 v_X+\epsilon  v_X+3 v_{\theta }\right)\right) \kappa }{12 \Delta ^{7/2}}-\frac{5 \left(\xi ^2 \left(-27 v_X+5
\Delta  \epsilon ^2 v_X+27 v_{\theta }\right)\right) \kappa ^2}{288 \Delta ^7}+O[\kappa ]^3\)
\end{align} \tlag{Yilin.24} \eeq


\beq \bgn
\gamma_{xx} &= \noindent\(-\frac{4 \epsilon  \xi  \left(-6 \Delta ^{5/2}+\epsilon  \kappa  \xi \right)}{36 \Delta ^3-9 \sqrt{\Delta } \epsilon  \kappa  \xi }+\frac{1}{9}
\kappa  \xi  \left(\frac{18}{8 \Delta ^{7/2}-15 \kappa  \xi }+\frac{\epsilon ^2 \kappa  \xi }{-4 \Delta ^6+\Delta ^{7/2} \epsilon  \kappa  \xi }\right)
v_z+\frac{19 \epsilon  \kappa  \xi  \left(-6 \Delta ^{5/2}+\epsilon  \kappa  \xi \right) v_{\theta }}{-144 \Delta ^6+36 \Delta ^{7/2} \epsilon  \kappa
 \xi }\) \\
&= \noindent\(\frac{2 \epsilon  \xi }{3 \sqrt{\Delta }}+\frac{\left(4 \sqrt{\Delta } \epsilon ^2 \xi ^2+18 \xi  v_z+57 \epsilon  \xi  v_{\theta }\right)
\kappa }{72 \Delta ^{7/2}}+\left(\frac{\epsilon ^3 \xi ^3}{72 \Delta ^{11/2}}-\frac{\left(-135+8 \Delta  \epsilon ^2\right) \xi ^2 v_z}{288 \Delta
^7}+\frac{19 \epsilon ^2 \xi ^2 v_{\theta }}{288 \Delta ^6}\right) \kappa ^2+O[\kappa ]^3\)
\end{align} \tlag{Yilin.25} \eeq


\beq \bgn
\gamma_{\theta\theta} &= \noindent\(-\frac{4 \epsilon  \xi  \left(-12 \Delta ^{5/2}+\epsilon  \kappa  \xi \right)}{36 \Delta ^3-9 \sqrt{\Delta } \epsilon  \kappa  \xi }+\frac{19
\epsilon ^2 \kappa ^2 \xi ^2 v_X}{36 \left(4 \Delta ^6-\Delta ^{7/2} \epsilon  \kappa  \xi \right)}+\kappa  \xi  \left(\frac{23 \epsilon }{36 \Delta
^{7/2}}+\frac{2}{8 \Delta ^{7/2}-15 \kappa  \xi }+\frac{34 \epsilon }{36 \Delta ^{7/2}-9 \Delta  \epsilon  \kappa  \xi }\right) v_z\) \\
&= \noindent\(\frac{4 \epsilon  \xi }{3 \sqrt{\Delta }}+\left(\frac{2 \epsilon ^2 \xi ^2}{9 \Delta ^3}+\frac{(3+19 \epsilon ) \xi  v_z}{12 \Delta ^{7/2}}\right)
\kappa +\left(\frac{\epsilon ^3 \xi ^3}{18 \Delta ^{11/2}}+\frac{19 \epsilon ^2 \xi ^2 v_X}{144 \Delta ^6}+\frac{\left(135+68 \Delta  \epsilon ^2\right)
\xi ^2 v_z}{288 \Delta ^7}\right) \kappa ^2+O[\kappa ]^3\)
\end{align} \tlag{Yilin.26} \eeq


\beq \bgn
\gamma_{x\theta} = \gamma_{\theta x} &= \noindent\(\frac{4 \epsilon ^2 \kappa  \xi ^2}{36 \Delta ^3-9 \sqrt{\Delta } \epsilon  \kappa  \xi }+\left(\frac{2}{15}-\frac{1}{2 \Delta }-\frac{3
\epsilon  \kappa  \xi }{8 \Delta ^{7/2}}+\frac{16}{15 \left(-8+\frac{15 \kappa  \xi }{\Delta ^{7/2}}\right)}+\frac{1}{2 \Delta -\frac{\epsilon  \kappa
 \xi }{2 \Delta ^{3/2}}}\right) v_z\) \\
&= \noindent\(\left(\frac{\epsilon ^2 \xi ^2}{9 \Delta ^3}-\frac{(\xi +\epsilon  \xi ) v_z}{4 \Delta ^{7/2}}\right) \kappa +\left(\frac{\epsilon ^3
\xi ^3}{36 \Delta ^{11/2}}+\frac{\left(-15+\Delta  \epsilon ^2\right) \xi ^2 v_z}{32 \Delta ^7}\right) \kappa ^2+O[\kappa ]^3\)
\end{align} \tlag{Yilin.27} \eeq

As we could see, $\gamma_{\alpha\beta}$ is a symmetric matrix. %Finally, we have obtained the all components of $\gamma_{\alpha\beta}$.
$$ \noindent\(\pmb{\gamma _{\text{}}=\left(
\begin{array}{ccc}
 \gamma _{\text{zz}} & \gamma _{\text{zx}} & \gamma _{\text{z$\theta $}} \\
 \gamma _{\text{xz}} & \gamma _{\text{xx}} & \gamma _{\text{x$\theta $}} \\
 \gamma _{\text{$\theta $z}} & \gamma _{\text{$\theta $x}} & \gamma _{\theta \theta } \\
\end{array}
\right)}\) $$
In addition, only three diagonal elements, namely $\gamma_{zz}, \gamma_{xx}, \gamma_{\theta\theta}$ have zero-order term of $\kappa$, which describes the compliance. In genenral, this parameter is about $10^{-4} \sim 10^{-3}$. %Hence we just consider the first-order correction in the following calculations. \\
Right now, we could write the equation of motion as:
%Up to now, we have elucidated the drift force. Indeed, there exist the external potential (like gravity) and white noises at the same time.
\beq \dot{v}_i = \frac{F_i}{m} - \gamma_{ij} v_j + \eta_i \tlag{Yilin.28} \eeq
where $F_i$ contains the force due to external potentials (gravity and buoyant force), as well as the spurious drift force, which originates from the derivative of $\gamma_{\alpha\beta}$, equal to $\pder[\gamma_{\alpha\beta}]{v_\beta} = \Gamma_{\alpha\beta\beta}$. See eqs \ref{Yilin.16}, we could easily obtain non-zero components:
$$ \bgn
\pder[\gamma_{zz}]{v_z} &= \Gamma_{zzz} = \frac{42 \kappa  \xi }{8 \Delta ^{9/2}-15 \Delta  \kappa  \xi } \approx \frac{21 \kappa  \xi }{4 \Delta ^{9/2}}+\frac{315 \kappa ^2 \xi ^2}{32 \Delta ^8} \\ 
\pder[\gamma_{zx}]{v_x} &= \Gamma_{zxx} = \frac{2 \kappa  \xi }{15 \kappa  \xi -8 \Delta ^{7/2}} \approx -\frac{\kappa  \xi }{4 \Delta ^{7/2}}-\frac{15 \kappa ^2 \xi ^2}{32 \Delta ^7} \\
\pder[\gamma_{z\theta}]{v_\theta} &= \Gamma_{z\theta\theta} = \frac{2 \kappa  \xi }{15 \kappa  \xi -8 \Delta ^{7/2}} \approx -\frac{\kappa  \xi }{4 \Delta ^{7/2}}-\frac{15 \kappa ^2 \xi ^2}{32 \Delta ^7} \\
\end{align} $$
Other $\Gamma_{\alpha\beta\beta}$s are all equal to 0. Thus there is only a spurious force on $\Delta$ direction.
%$$ \dv{t} \left(\begin{array}{c}v_Z \\v_X \\v_\Theta \end{array}\right) = M^{-1} \left(\begin{array}{c} F_{1hZ} - \nabla_Z \phi \\F_{1hX} \\F_{1h\Theta}\end{array}\right) + T \left(\begin{array}{c}\Gamma_{Z\beta\beta} \\\Gamma_{X\beta\beta}=0 \\\Gamma_{\Theta\beta\beta}=0\end{array}\right) + \left(\begin{array}{c}\eta_Z \\\eta_X \\\eta_\Theta\end{array}\right)$$

To be exact: 
%Since three components are coupled with each other, the equation would turn to (For the spurious force, see subsection \ref{sdrift} )
\beq \left(\begin{array}{c}\dot{v}_\Delta \\\dot{v}_X \\\dot{v}_\Theta\end{array}\right) = \left(\begin{array}{c}F_\Delta \\ F_X \\ F_\Theta\end{array}\right) - \left(
\begin{array}{ccc}
 \gamma _{\text{zz}} & \gamma _{\text{zx}} & \gamma _{\text{z$\theta $}} \\
 \gamma _{\text{xz}} & \gamma _{\text{xx}} & \gamma _{\text{x$\theta $}} \\
 \gamma _{\text{$\theta $z}} & \gamma _{\text{$\theta $x}} & \gamma _{\theta \theta } \\
\end{array}
\right)\left(\begin{array}{c}v_\Delta \\v_X \\v_\Theta\end{array}\right) + \left(\begin{array}{c}\eta_\Delta \\\eta_X \\\eta_\Theta\end{array}\right) \tlag{Yilin.29} \eeq
As for the noise correlator, it has been shown in the eq. \ref{David.25}: 
$$\llang \eta_\alpha (t) \eta_\beta(t^\prime) \rrang = 2T \gamma_{\alpha \beta} \delta(t-t^\prime) = 2T \left[ \lambda_{\alpha\beta}(\mathbf{X}) + \Gamma_{\alpha\beta\gamma}(\mathbf{X}) V_\gamma \right] \delta(t-t^\prime)$$
So we'd like to find the expression of $\gamma^{1/2}$. Suppose that 
\beq \gamma = \Psi + \kappa \Phi + O[\kappa]^2 \tlag{Yilin.30} \eeq
where $\Psi$ is zero-order matrix of $\kappa $, and $\Phi $ the first-order one. Also, $\gamma^{1/2}$ show a form such as $\gamma^{1/2} \approx \psi + \kappa\chi$, 
$$ \gamma = \gamma^{1/2}\gamma^{1/2} = (\psi + \kappa\chi)(\psi + \kappa\chi) = \psi\psi + \kappa (\psi\chi + \chi\psi) + O[\kappa ]^2$$
we have $\Phi=\chi \psi +\psi \chi$, and \(\psi =\sqrt{\Psi }\). Note $\Psi$ is a symmetric matrix and all non-diagonal elements are equal to 0.
\beq \(\Psi =\left(
\begin{array}{ccc}
 \lambda_z & 0 & 0 \\
 0 & \lambda_x & 0 \\
 0 & 0 & \lambda_\theta \\
\end{array}
\right)\) = \left(\begin{array}{ccc}\frac{\xi }{\Delta ^{3/2}} & 0 & 0 \\0 & \frac{2 \epsilon  \xi }{3 \sqrt{\Delta }} & 0 \\0 & 0 & \frac{4 \epsilon  \xi }{3 \sqrt{\Delta }}\end{array}\right) \tlag{Yilin.31} \eeq
Also, $\chi$ is a symmetric matrix, with diagonal components as:
\beq \bgn
\chi_{zz} &= \noindent\(\frac{3 \xi  \left(5 \xi +14 \sqrt{\Delta } v_z\right)}{16 \Delta ^5 \sqrt{\frac{\xi }{\Delta ^{3/2}}}}\) \\
\chi_{xx} &= \noindent\(\frac{\xi  \left(18 v_z+\epsilon  \left(4 \sqrt{\Delta } \epsilon  \xi +57 v_{\theta }\right)\right)}{48 \sqrt{6} \Delta ^{7/2} \sqrt{\frac{\epsilon
 \xi }{\sqrt{\Delta }}}}\) \\
\chi_{\theta\theta} &= \noindent\(\frac{\xi  \left(8 \sqrt{\Delta } \epsilon ^2 \xi +(9+57 \epsilon ) v_z\right)}{48 \sqrt{3} \Delta ^{7/2} \sqrt{\frac{\epsilon  \xi }{\sqrt{\Delta
}}}}\)
\end{align} \tlag{Yilin.32} \eeq
and non-diagonal elements
\beq \bgn
\chi_{zx} = \chi_{xz} &= \noindent\(\frac{\xi  \left(-3 v_X+(3+\epsilon ) v_{\theta }\right)}{4 \Delta ^{7/2} \left(3 \sqrt{\frac{\xi }{\Delta ^{3/2}}}+\sqrt{6} \sqrt{\frac{\epsilon
 \xi }{\sqrt{\Delta }}}\right)}\) \\
\chi_{z\theta} = \chi_{\theta z} &= -\frac{\xi \left(3 v_{\theta}+(\epsilon -3) v_X\right)}{4 \Delta ^{7/2} \left(3 \sqrt{\frac{\xi }{\Delta ^{3/2}}}+2 \sqrt{3} \sqrt{\frac{\xi  \epsilon }{\sqrt{\Delta }}}\right)} \\
\chi_{x\theta} = \chi_{\theta x} &= \noindent\(\frac{\xi  \left(4 \sqrt{\Delta } \epsilon ^2 \xi -9 (1+\epsilon ) v_z\right)}{12 \sqrt{3} \left(2+\sqrt{2}\right) \Delta ^{7/2} \sqrt{\frac{\epsilon
 \xi }{\sqrt{\Delta }}}}\)
\end{align} \tlag{Yilin.33} \eeq


\ssc{Mass vector}
We always regarded the mass as 1 in the previous parts, while it should be carefully distinguished later. For $z,x$ components, $m_\alpha = m = \pi r^2 \rho$ namely the mass of the column (per unit length). However, $m_\Theta = mr^2 /2$ refers to the moment of inertia. Taking that into account, we compare these two versions (Note we re-write $U_\alpha$ as $F_\alpha$ on the left side)

\begin{center}
\begin{tabular}{c|c}
\hline
$m_\alpha = 1$ & $m_\alpha = (m,m,mr^2/2)$ \\
\hline
$dX_\alpha = V_\alpha dt$ & $\dot{x}_\alpha = v_\alpha \\
$dV_\alpha = - F_\alpha dt - \nabla_\alpha \phi(\mathbf{X}) dt$ & $m_\alpha \cdot \dot{v}_\alpha = F_{h\alpha}(\bbf{v},\dot{\bbf{v}},\bbf{x}) - \nabla_\alpha \phi(\bbf{x})$ \\
$F_\alpha = \gamma_{\alpha\beta} V_\beta = \lambda_{\alpha\beta} V_\beta + \Gamma_{\alpha\beta\gamma} V_\beta V_\gamma$ & $F_{h\alpha}(\bbf{v},\dot{\bbf{v}},\bbf{x}) = F_{1h\alpha}(\bbf{v},\bbf{x}) + F_{2h\alpha\beta}(\bbf{x}) \dot{v}_\beta $ \\
$$ & $M_{\alpha\beta} = \delta_{\alpha\beta} \cdot m_\alpha - F_{2h\alpha\beta}(\bbf{x})$ \\
$$ & $M_{\alpha\beta} \dot{v}_\beta = F_{1h\alpha}(\bbf{v},\bbf{x}) - \nabla_\alpha \phi(\bbf{x}) $ \\
%$P_{eq} (\mathbf{X},\mathbf{V}) = \frac{1}{\bar{Z}} \exp \left( - \frac{\beta \mathbf{V}^2}{2} - \beta \phi(\mathbf{X}) \right)$ & $P_0 = \exp \left[ -\beta \sum_\alpha \frac{m_\alpha}{2} v_\alpha^2 - \beta \phi(\bbf{x}) \right] $ \\
%$\frac{dV_\alpha}{dt} = -F_\alpha - \pder[\phi(\mathbf{X})]{X_\alpha} + T \pder[\gamma_{\alpha\beta}]{V_\beta} + \eta_\alpha(t)$ & $\dot{v}_\alpha (t) = M_{\alpha\beta}^{-1} \left[ F_{1h\beta} (\bbf{v},\bbf{x}) - \nabla_\beta \phi(\bbf{x}) \right] + TU_\alpha + \eta_\alpha $ \\
%$\pder[P]{t} = \pder{V_\alpha} \left[ T \gamma_{\alpha\beta} \pder[P]{V_\beta} + U_\alpha P + \pder[\phi]{X_\alpha} P \right] - \pder{X_\alpha} V_\alpha P$ & $\pder{v_\alpha} \left\{ T \gamma_{\alpha\beta} \pder[P]{v_\alpha} - \left[ M_{\alpha\beta}^{-1} \left( F_{1h\beta}(\bbf{v},\bbf{x}) - \nabla_\beta \phi(\bbf{x}) \right) - TU_\alpha \right] P \right\} - \pder{x_\alpha} (v_\alpha P)$ \\
%$F_\alpha = \gamma_{\alpha\beta} V_\beta = \lambda_{\alpha\beta} V_\beta + \Gamma_{\alpha\beta\gamma} V_\beta V_\gamma$ & $\beta v_\alpha \nabla_\alpha \phi P + \nabla_\beta \phi(\bbf{x}) \pder{v_\alpha} \left[ M_{\alpha\beta}^{-1} P \right] = 0 $ \\
\hline
\end{tabular}
\end{center}

Since $F_{h\alpha}(\bbf{v},\dot{\bbf{v}},\bbf{x}) = F_{1h\alpha}(\bbf{v},\bbf{x}) + F_{2h\alpha\beta}(\bbf{x}) \dot{v}_\beta $, we could extract $F_{1h\alpha}(\bbf{v},\bbf{x})$ and $F_{2h\alpha\beta}(\bbf{x})$ by %Along the perpendicular direction, 
\beq - \frac{F_{hZ}}{m_Z} = - \dot{v}_z = - \ddot{\Delta} = a_1 \frac{\dot{\Delta}}{\Delta^{3/2}} + a_2 \frac{\dot{\Delta}^2}{\Delta^{9/2}} + a_3 \frac{\dot\Theta^2}{\Delta^{7/2}} + a_3 \frac{\dot{X}^2}{\Delta^{7/2}} + a_4 \frac{\dot\Theta \dot{X}}{\Delta^{7/2}} + a_5 \frac{\ddot\Delta}{\Delta^{7/2}} + a_6 \tlag{Yilin.34} \eeq
thus
\beq \bgn
F_{1hZ} &= -m_Z \left( a_1 \frac{\dot{\Delta}}{\Delta^{3/2}} + a_2 \frac{\dot{\Delta}^2}{\Delta^{9/2}} + a_3 \frac{\dot\Theta^2}{\Delta^{7/2}} + a_3 \frac{\dot{X}^2}{\Delta^{7/2}} + a_4 \frac{\dot\Theta \dot{X}}{\Delta^{7/2}} + a_6 \right) \\
F_{2hZZ} &= - \frac{m_Z a_5}{\Delta^{7/2}} \tens F_{2hZX} = 0 \tens F_{2hZ\Theta} = 0
\end{align} \tlag{Yilin.35} \eeq
Similarly, there are cross terms for $X, \Theta$ components
\beq - \frac{F_{hX}}{m_X} = - \dot{v}_x = - \textcolor{red}{\ddot{X}} = b_1 \frac{\dot{X}}{\sqrt\Delta} + b_2  \frac{\dot\Delta \dot{X}}{\Delta^{7/2}} + b_3 \frac{\dot\Delta \dot\Theta}{\Delta^{7/2}} + b_4 \frac{\textcolor{blue}{\ddot{\Theta}}}{\Delta^{5/2}} + b_5 \frac{\textcolor{red}{\ddot{X}}}{\Delta^{5/2}} + b_6 \tlag{Yilin.36} \eeq
\beq \bgn
F_{1hX} &= -m_X \left( b_1 \frac{\dot{X}}{\sqrt\Delta} + b_2  \frac{\dot\Delta \dot{X}}{\Delta^{7/2}} + b_3 \frac{\dot\Delta \dot\Theta}{\Delta^{7/2}} + b_6 \right) \\
F_{2hXZ} &=0 \tens F_{2hXX} = - \frac{m_X b_5}{\Delta^{5/2}} \tens F_{2hX\Theta} = - \frac{m_X b_4}{\Delta^{5/2}}
\end{align} \tlag{Yilin.37} \eeq
and
\beq - \frac{F_{h\Theta}}{m_\Theta} = - \dot{v}_\theta = - \textcolor{blue}{\ddot{\Theta}} = c_1 \frac{\dot\Theta}{\sqrt\Delta} + c_2 \frac{\dot\Delta \dot\Theta}{\Delta^{7/2}} + c_3 \frac{\dot\Delta \dot{X}}{\Delta^{7/2}} + c_4 \frac{\textcolor{red}{\ddot{X}}}{\Delta^{5/2}} + c_5 \frac{\textcolor{blue}{\ddot{\Theta}}}{\Delta^{5/2}} + c_6 \tlag{Yilin.38} \eeq
\beq \bgn
F_{1h\Theta} &= -m_\Theta \left( c_1 \frac{\dot\Theta}{\sqrt\Delta} + c_2 \frac{\dot\Delta \dot\Theta}{\Delta^{7/2}} + c_3 \frac{\dot\Delta \dot{X}}{\Delta^{7/2}} + c_6 \right) \\
F_{2h\Theta Z} &=0 \tens F_{2h\Theta X} = - \frac{m_\Theta c_4}{\Delta^{5/2}} \tens F_{2h\Theta\Theta} = - \frac{m_\Theta c_5}{\Delta^{5/2}}
\end{align} \tlag{Yilin.39} \eeq
We pose that $M_{\alpha\beta} = \delta_{\alpha\beta} m_\alpha - F_{2h\alpha\beta}(\bbf{x})$, hence ($m_X = m, m_\Theta = m r^2/2$) 
\beq \bgn
M_{ZZ} &= m_Z + \frac{m_Z a_5}{\Delta^{5/2}} \\
M_{XX} &= m_X + \frac{m_X b_5}{\Delta^{5/2}} \tens M_{X\Theta} = \frac{m_X b_4}{\Delta^{5/2}} \\
M_{\Theta X} &= \frac{m_\Theta c_4}{\Delta^{5/2}} \tens M_{\Theta \Theta} = m_\Theta + \frac{m_\Theta c_5}{\Delta^{5/2}}
\end{align} \tlag{Yilin.40} \eeq
We know that $a_5 = -\frac{15\kappa\xi}{8}$, $b_4 = \frac{\kappa\xi\veps}{12}$, $b_5 = - \frac{\kappa\xi\veps}{12}$, $c_4 = \frac{\kappa\xi\veps}{6}$, $c_5 = - \frac{\kappa\xi\veps}{6}$, so
\beq M = \left(
\begin{array}{ccc}
 m_z-\frac{15 \kappa  \xi  m_z}{8 \Delta ^{5/2}} & 0 & 0 \\
 0 & m_X-\frac{\kappa  \xi  \epsilon  m_X}{12 \Delta ^{5/2}} & \frac{\kappa  \xi  \epsilon  m_X}{12 \Delta ^{5/2}} \\
 0 & \frac{\kappa  \xi  \epsilon  m_{\theta }}{6 \Delta ^{5/2}} & m_{\theta }-\frac{\kappa  \xi  \epsilon  m_{\theta }}{6 \Delta ^{5/2}} \\
\end{array}
\right) \tlag{Yilin.41} \eeq
and its inverse matrix 
\beq M^{-1} = \left(
\begin{array}{ccc}
 \frac{1}{m_z-\frac{15 \kappa  \xi  m_z}{8 \Delta ^{5/2}}} & 0 & 0 \\
 0 & \frac{12 \Delta ^{5/2}-2 \kappa  \xi  \epsilon }{12 \Delta ^{5/2} m_X-3 \kappa  \xi  \epsilon  m_X} & \frac{\kappa  \xi  \epsilon }{3 m_{\theta } \left(\kappa  \xi  \epsilon -4 \Delta ^{5/2}\right)} \\
 0 & \frac{2 \kappa  \xi  \epsilon }{3 m_X \left(\kappa  \xi  \epsilon -4 \Delta ^{5/2}\right)} & \frac{12 \Delta ^{5/2}-\kappa  \xi  \epsilon }{12 \Delta ^{5/2} m_{\theta }-3 \kappa  \xi  \epsilon  m_{\theta }} \\
\end{array}
\right) \tlag{Yilin.42} \eeq
with the approximation expressed by the series of $\kappa$:
\beq M^{-1}_{app} \approx \left(
\begin{array}{ccc}
 \frac{1}{m_z}+\frac{15 \kappa  \xi }{8 \Delta ^{5/2} m_z}+\frac{225 \kappa ^2 \xi ^2}{64 \Delta ^5 m_z}   & 0 & 0 \\
 0 & \frac{1}{m_X}+\frac{\kappa  \xi  \epsilon }{12 \Delta ^{5/2} m_X}+\frac{\kappa ^2 \xi ^2 \epsilon ^2}{48 \Delta ^5 m_X}   & -\frac{\kappa  (\xi  \epsilon )}{12 \left(\Delta ^{5/2} m_{\theta }\right)}-\frac{\kappa ^2 \left(\xi ^2 \epsilon ^2\right)}{48 \left(\Delta ^5 m_{\theta }\right)}   \\
 0 & -\frac{\kappa  (\xi  \epsilon )}{6 \left(\Delta ^{5/2} m_X\right)}-\frac{\kappa ^2 \left(\xi ^2 \epsilon ^2\right)}{24 \left(\Delta ^5 m_X\right)}   & \frac{1}{m_{\theta }}+\frac{\kappa  \xi  \epsilon }{6 \Delta ^{5/2} m_{\theta }}+\frac{\kappa ^2 \xi ^2 \epsilon ^2}{24 \Delta ^5 m_{\theta }}   \\
\end{array}
\right) \tlag{Yilin.43} \eeq
Only taking the first-order correction, we could verify
$$ M \cdot M^{-1}_{app} = \left( \begin{array}{ccc} 1-\frac{225 \kappa ^2 \xi ^2}{64 \Delta ^5} & 0 & 0 \\ 0 & 1-\frac{\kappa ^2 \xi ^2 \epsilon ^2}{48 \Delta ^5} & \frac{\kappa ^2 \xi ^2 \epsilon ^2 m_X}{48 \Delta ^5 m_{\theta }} \\ 0 & \frac{\kappa ^2 \xi ^2 \epsilon ^2 m_{\theta }}{24 \Delta ^5 m_X} & 1-\frac{\kappa ^2 \xi ^2 \epsilon ^2}{24 \Delta ^5} \\ \end{array} \right) \approx \left(\begin{array}{ccc}1 & 0 & 0 \\0 & 1 & 0 \\0 & 0 & 1\end{array}\right)$$
% Or taking the second-order term, we would get
%$$ M \cdot M^{-1}_{app} = \left( \begin{array}{ccc} 1-\frac{3375 \kappa ^3 \xi ^3}{512 \Delta ^{15/2}} & 0 & 0 \\ 0 & 1-\frac{\kappa ^3 \xi ^3 \epsilon ^3}{192 \Delta ^{15/2}} & \frac{\kappa ^3 \xi ^3 \epsilon ^3 m_X}{192 \Delta ^{15/2} m_{\theta }} \\ 0 & \frac{\kappa ^3 \xi ^3 \epsilon ^3 m_{\theta }}{96 \Delta ^{15/2} m_X} & 1-\frac{\kappa ^3 \xi ^3 \epsilon ^3}{96 \Delta ^{15/2}} \\ \end{array} \right) \approx \left(\begin{array}{ccc}1 & 0 & 0 \\0 & 1 & 0 \\0 & 0 & 1\end{array}\right)$$







%\newpage
\sct{Numerical Practice}
\ssc{JPCB2014: Discrete-Time Langevin Integration}
\tit{J. Phys. Chem. B, 2014, 118, 6466-6474, one article about Discrete-Time Langevin Integration. For multiple dimensions, see its Support Information:}\\
https://pubs.acs.org/doi/suppl/10.1021/jp411770f/suppl\_file/jp411770f\_si\_001.pdf

Consider a Langevin equation
$$ dv = \frac{f(t)}{m} dt - \gamma v dt + \sqrt{\frac{2\gamma}{\beta m}} dW(t) $$
we have 
\begin{itemize}
  \item Ornstein-Uehlenbeck operator for stochastic thermalization: $\scr{L}_o = -\gamma \pder{v} v - \frac{\gamma}{\beta m} \pdv[2]{}{x}$
  \item Deterministic Newtonian evolutions: $\scr{L}_v = \frac{f}{m}$, $\scr{L}_r = v \pder{r}$
  \item Hamiltonian: $\exp(\scr{L}_h \Delta t) \scr{H}(n) = \scr{H}(n+1)$
\end{itemize}
where $n$ is the time step index and $t=n\Delta t$.

For this operator splitting, a single update step that advances the simulation clock by $\Delta t$ is given explicitly by
$$ \agn{
& \bbf{v}\left( n+ \frac{1}{4}\right) = \sqrt{a} \cdot \bbf{v}(n) + \left[ \frac{1}{\beta} (\bbf{1} - \bbf{a}) \cdot \bbf{m}^{-1} \right]^{1/2} \cdot \bbf{N}^{+} (n) \\
& \fives \bbf{v} \left( n+ \frac{1}{2}\right) = \bbf{v} \left( n+ \frac{1}{4}\right) + \frac{\Delta t}{2} \bbf{b} \cdot \bbf{m}^{-1} \cdot \bbf{f}(n) \\
& \fives \fives \bbf{r} \left( n+ \frac{1}{2}\right) = \bbf{r}(n) + \frac{\Delta t}{2} \bbf{b} \cdot \bbf{v} \left( n+ \frac{1}{2}\right) \\
& \fives \fives \fives \scr{H}(n) \to \scr{H}(n+1) \\
& \fives \fives \bbf{r} \left( n+1\right) = \bbf{r} \left( n+ \frac{1}{2}\right) + \frac{\Delta t}{2} \bbf{b} \cdot \bbf{v} \left( n+ \frac{1}{2}\right) \\
& \fives v\left( n+ \frac{3}{4}\right) = v\left( n+ \frac{1}{2}\right) + \frac{\Delta t}{2} \bbf{b} \cdot \bbf{m}^{-1} \cdot \bbf{f}(n+1) \\
& \bbf{v}\left( n+1 \right) = \sqrt{a} \cdot \bbf{v}\left( n+ \frac{3}{4}\right) + \left[ \frac{1}{\beta} (\bbf{1} - \bbf{a}) \cdot \bbf{m}^{-1} \right]^{1/2} \cdot \bbf{N}^{-} (n+1) \\
} $$
where $a_{ij} = \delta_{ij} \exp(-\gamma_i \Delta t)$, $\scr{N}^\pm$ are independent normally distributed random variables with zero mean and unit variance, $b_{ij} = \delta_{ij} \sqrt{\frac{2}{\gamma_i \Delta t} \tanh \frac{\gamma_i \Delta t}{2}}$



\ssc{Update $\gamma_\rm{eff}}
We have obtained the $\gamma$ matrix in the subsection \ref{gamma}, without mass vector. Here we update the effective matrix $\gamma_\mrm{eff}$ with $M^{-1}$, starting from 
$$ \begin{align}
m_\al \cdot \dot{v}_\al &= F_\al (t) - m_\alpha \cdot \gamma_{\alpha\beta} v_\bt = \left[ F_{1\al}(\bbf{x}) + F_{2\ab}(\bbf{x}) \dot{v}_\bt \right] - m_\al \cdot \gamma_{\ab} v_\bt \\
m_{\al} \cdot \dot{v}_\al &- F_{2\ab}(\bbf{x}) \dot{v}_\bt = \left( m_\al \cdot \delta_{\ab} - F_{2\ab}(\bbf{x}) \right) \dot{v}_\beta = F_{1\al}(\bbf{x}) - m_\al \cdot \gamma_{\ab} v_\beta \\
\dot{v}_\bt &= \left( m_\al \cdot \delta_{\ab} - F_{2\ab}(\bbf{x}) \right)^{-1} \left( F_{1\al}(\bbf{x}) - m_\al \cdot \gamma_{\ab} v_\beta \right) = M_{\ab}^{-1} \left( F_{1\al}(\bbf{x}) - m_\al \cdot \gamma_{\ab} v_\beta \right)
\end{align} $$
Note that the $\gamma$ matrix above only contains terms about first derivatives %does not contain coefficients $(a/b/c)_{5/6}$
$$ \agn{
\gamma_{Z\beta} v_\beta &= a_1 \frac{\dot{\Delta}}{\Delta^{3/2}} + a_2 \frac{\dot{\Delta}^2}{\Delta^{9/2}} + a_3 \frac{\dot\Theta^2}{\Delta^{7/2}} + a_3 \frac{\dot{X}^2}{\Delta^{7/2}} + a_4 \frac{\dot\Theta \dot{X}}{\Delta^{7/2}}\\
\gamma_{X\beta} v_\beta &= b_1 \frac{\dot{X}}{\sqrt\Delta} + b_2  \frac{\dot\Delta \dot{X}}{\Delta^{7/2}} + b_3 \frac{\dot\Delta \dot\Theta}{\Delta^{7/2}} \\
\gamma_{\Theta\beta} v_\beta &= c_1 \frac{\dot\Theta}{\sqrt\Delta} + c_2 \frac{\dot\Delta \dot\Theta}{\Delta^{7/2}} + c_3 \frac{\dot\Delta \dot{X}}{\Delta^{7/2}} 
} $$
To avoid the possible confusion, we write $\gamma^\ast$ below. Consider $dv = \frac{f(t)}{m} dt - \gamma v dt + \sqrt{\frac{2\gamma}{\beta m}} dW $, we have $\gamma_\mrm{eff}$ :
\eq{\gamma_\mrm{eff} = M_{\ab}^{-1} \cdot \left(\begin{array}{ccc}m_Z & 0 & 0 \\0 & m_X & 0 \\0 & 0 & m_\Theta \end{array}\right) \cdot \gamma^\ast_{\ab} \tlag{Yilin.44} }
Surprisingly, we recover almost the same $\gamma_{\ab}$ shown previously
$$ \agn{
\gamma_{\mrm{eff},zz} &= \frac{\xi }{\Delta ^{3/2}} +\kappa  \left(\frac{15 \xi ^2}{8 \Delta ^4}+\frac{21 \xi  v_z}{4 \Delta ^{9/2}}\right) +O\left(\kappa ^2\right) \\
\gamma_{\mrm{eff},xx} &= \frac{2 \xi  \epsilon }{3 \sqrt{\Delta }}+\frac{\kappa  \xi  \left(4 \sqrt{\Delta } \xi  \epsilon ^2+18 v_z+57 \epsilon  v_{\theta }\right)}{72 \Delta ^{7/2}}+O\left(\kappa ^2\right) \\
\gamma_{\mrm{eff},\theta\theta} &= \frac{4 \xi  \epsilon }{3 \sqrt{\Delta }}+\frac{\kappa  \xi  \left(8 \sqrt{\Delta } \xi  \epsilon ^2+57 \epsilon  v_z+9 v_z\right)}{36 \Delta ^{7/2}}+O\left(\kappa ^2\right) \\
\gamma_{\mrm{eff},zx} &= \frac{\kappa  \xi  \left((\epsilon +3) v_{\theta }-3 v_X\right)}{12 \Delta ^{7/2}} + O\left(\kappa ^2\right) \\
\gamma_{\mrm{eff},z\theta} &= -\frac{\kappa  \left(\xi  \left(3 v_{\theta }+(\epsilon -3) v_X\right)\right)}{12 \Delta ^{7/2}}+O\left(\kappa ^2\right) \\
\gamma_{\mrm{eff},x\theta} &= -\frac{\kappa  \left(\xi  \left(16 \Delta ^3 \xi  \epsilon ^2+36 \Delta ^{5/2} (\epsilon +1) v_z\right)\right)}{144 \Delta ^6}+O\left(\kappa ^2\right)
} $$




\ssc{Possible modification}
Here we consider the following approaches for the sake of better numerical practices
\begin{itemize}
  \item Solve Langevin equation numerically or solve Fokker-Planck equation directly;
  \item Add second derivatives or not; ($b_4 = b_5 = c_4 = c_5 = 0$)
  \item Ignore rotation and fix $\Theta = 0$
  \item Modify Gaussian random force with "Heaviside" function or not; to avoid sudden collapse $z<0$
  \item How to determine the initial $z$ value near the equilibrium position;
  \item Add Coulomb interaction or not; ($\frac{Q^2}{4\pi\veps z} \exp(-\lambda_D/z)$)
  \item Diagonalize M matrix or not? physical meaning?
  \item Consider $\kappa$ correction to which order?
\end{itemize}



\nocite{EINAV2010,Havrylchyk2018} 

\bibliographystyle{aer}
\bibliography{wp_ref}
\end{document}

