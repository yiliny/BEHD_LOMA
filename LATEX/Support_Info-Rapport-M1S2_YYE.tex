%%%%%%%%%%%%%%%%%%%%%%%%%%%%%%%%%%%%%%%%%%
% Internship Report template
% Chemistry department 
% Version 1.1 (15/02/14)
%%%%%%%%%%%%%%%%%%%%%%%%%%%%%%%%%%%%%%%%%%

%----------------------------------------------------------------------------------------
%	PACKAGES AND OTHER DOCUMENT CONFIGURATIONS
%----------------------------------------------------------------------------------------

\documentclass[fleqn,10pt]{InternshipReport_SI-ENS-PSL}

\setlength{\columnsep}{0.55cm} % Distance between the two columns of text
\setlength{\fboxrule}{0.75pt} % Width of the border around the abstract

\definecolor{color1}{RGB}{60,23,61} % Color of the article title and sections
\definecolor{color2}{RGB}{20,00,20} % Color of the boxes behind the abstract and headings


\usepackage{amsthm,amsmath,amssymb}
\usepackage{mathrsfs}
\usepackage{physics}
\usepackage{cancel} %用于在偏微分符号上画斜线
\usepackage{ulem} %波浪线, 双下划线
\usepackage{newcommand_yye}


%%%%% LOMA图标
%%左上角?右下角?
%\includegraphics[width=0.05\linewidth]{EmetBrownVF.png}


%%%%% 流程图用tikz
\usepackage{tikz}
\usetikzlibrary{positioning, shapes.geometric}
\usetikzlibrary{graphs, positioning, quotes, shapes.geometric}



%%%%% 页眉页脚
\pagestyle{fancy}
%\fancyhf{}
\fancyhead[RE,LO]{Yilin YE} % Right on Even page, Left on Odd page
%\fancyfoot[RE,LO]{NMR@ENS M1} % 每页左下角
%\fancyfoot[LE,RO]{Yilin YE} % 每页右下角
%\rhead{\includegraphics[width=0.8cm]{EmetBrownVF.png}} % 每页右上角,与页码冲突
%\lhead{\includegraphics[width=0.8cm]{Universitat_Bordeaux_Logo.png}}
\lfoot{
\includegraphics[height=0.7cm]{Universitat_Bordeaux_Logo.png} \includegraphics[height=0.6cm]{LOMA-CNRS-logo.png}
\includegraphics[height=0.8cm]{cnrs_logo.png}}
\rfoot{
\includegraphics[height=0.5cm]{anr_logo.png} \includegraphics[height=0.8cm]{erc_logo_1.png}
\includegraphics[height=0.8cm]{EmetBrownVF.png}}



%----------------------------------------------------------------------------------------
%	ARTICLE INFORMATION
%----------------------------------------------------------------------------------------

\ReportTitle{Brownian Motion near a Soft Surface \\ \tit{Supporting Information}} % Article title
\Author{Yilin YE}
\Supervisor{Yacine AMAROUCHENE, David DEAN, Thomas SALEZ}
\Laboratory{Université de Bordeaux, CNRS, Laboratoire Ondes et Matière d'Aquitaine, UMR 5798, F-33405, Talence, France}

%\Keywords{Brownian motion --- soft surface --- Langevin equation --- noise correlator --- Fokker-Planck equation} 
%\Keywords{Keyword1 --- Keyword2 --- Keyword3 --- Keyword4 --- Keyword5} 
% Keywords - if you don't want any simply remove all the text between the curly brackets
%\newcommand{\keywordname}{Keywords} 
% Defines the keywords heading name

%----------------------------------------------------------------------------------------
%	ABSTRACT
%----------------------------------------------------------------------------------------



%----------------------------------------------------------------------------------------

\begin{document}

\flushbottom % Makes all text pages the same height

\maketitle % Print the title without abstract box

\thispagestyle{empty} % Removes page numbering from the first page

%----------------------------------------------------------------------------------------
%	ARTICLE CONTENTS
%----------------------------------------------------------------------------------------


%\section*{Introduction} % The \section*{} command stops section numbering
 %Here is the Supporting Information for the internship report "Brownian Motion near a Soft Surface".




%------------------------------------------------

\section*{Theoretical Analyses}

\ssc*{Situation of the problem}

The equations of motion are shown below \cite{JFM2015}
$$ \textcolor{red}{\ddot{X}_G} + \frac{2\varepsilon \xi}{3} \frac{\dot{X}_G}{\sqrt\Delta} + \frac{\textcolor{brown}{\kappa} \varepsilon \xi}{6} \left[ \frac{19}{4} \frac{\dot\Delta \dot{X}_G}{\Delta^{7/2}} - \frac{\dot\Delta \dot\Theta}{\Delta^{7/2}} + \frac{1}{2} \frac{\textcolor{blue}{\ddot{\Theta}} - \textcolor{red}{\ddot{X}_G}}{\Delta^{5/2}} \right] - \sqrt{\frac{\varepsilon}{2}} \sin\alpha = 0 $$

$$ \textcolor{green}{\ddot{\Delta}} + \xi \frac{\dot{\Delta}}{\Delta^{3/2}} + \frac{\textcolor{brown}{\kappa} \xi}{4} \left[ 21 \frac{\dot{\Delta}^2}{\Delta^{9/2}} - \frac{(\dot\Theta - \dot{X}_G)^2}{\Delta^{7/2}} - \frac{15}{2} \frac{\textcolor{green}{\ddot{\Delta}}}{\Delta^{7/2}} \right] + \cos\alpha = 0  $$

$$ \textcolor{blue}{\ddot{\Theta}} + \frac{4\veps\xi}{3} \frac{\dot\Theta}{\sqrt\Delta} + \frac{\textcolor{brown}{\kappa} \veps \xi}{3} \left[ \frac{19}{4} \frac{\dot\Delta \dot\Theta}{\Delta^{7/2}} - \frac{\dot\Delta \dot{X}_G}{\Delta^{7/2}} + \frac{1}{2} \frac{\textcolor{red}{\ddot{X}_G} - \textcolor{blue}{\ddot{\Theta}}}{\Delta^{5/2}} \right] = 0  $$
with $\alpha=0$ for a plan case.


For the convenience, we simplify all coefficients as:
$$ \textcolor{green}{\ddot{\Delta}} + a_1 \frac{\dot{\Delta}}{\Delta^{3/2}} + a_2 \frac{\dot{\Delta}^2}{\Delta^{9/2}} + a_3 \frac{\dot\Theta^2}{\Delta^{7/2}} + a_3 \frac{\dot{X}^2}{\Delta^{7/2}} + a_4 \frac{\dot\Theta \dot{X}}{\Delta^{7/2}} + a_5 \frac{\textcolor{green}{\ddot{\Delta}}}{\Delta^{7/2}} + a_6 = 0 $$

$$ \textcolor{red}{\ddot{X}_G} + b_1 \frac{\dot{X}}{\sqrt\Delta} + b_2  \frac{\dot\Delta \dot{X}}{\Delta^{7/2}} + b_3 \frac{\dot\Delta \dot\Theta}{\Delta^{7/2}} + b_4 \frac{\textcolor{blue}{\ddot{\Theta}}}{\Delta^{5/2}} + b_5 \frac{\textcolor{red}{\ddot{X}_G}}{\Delta^{5/2}} + b_6 = 0 $$

$$ \textcolor{blue}{\ddot{\Theta}} + c_1 \frac{\dot\Theta}{\sqrt\Delta} + c_2 \frac{\dot\Delta \dot\Theta}{\Delta^{7/2}} + c_3 \frac{\dot\Delta \dot{X}}{\Delta^{7/2}} + c_4 \frac{\textcolor{red}{\ddot{X}_G}}{\Delta^{5/2}} + c_5 \frac{\textcolor{blue}{\ddot{\Theta}}}{\Delta^{5/2}} + c_6 = 0 $$
with coefficients: $a_1 = \xi$, $a_2 = \frac{21\kappa\xi}{4}$, $a_3 = -\frac{\kappa\xi}{4}$, $a_4 = \frac{\kappa\xi}{2}$, $a_5 = -\frac{15\kappa\xi}{8}$, $a_6 = \cos(\alpha=0) = 1$; 
$b_1 = \frac{2\veps\xi}{3}$, $b_2 = \frac{19\kappa\xi\veps}{24}$, $b_3 = - \frac{\kappa\xi\veps}{6}$, $b_4 = \frac{\kappa\xi\veps}{12}$, $b_5 = - \frac{\kappa\xi\veps}{12}$, $b_6 = \sin(\alpha = 0) = 0$;
and $c_1 = \frac{4\veps\xi}{3}$, $c_2 = \frac{19\kappa\xi\veps}{12}$, $c_3 = - \frac{\kappa\xi\veps}{3}$, $c_4 = \frac{\kappa\xi\veps}{6}$, $c_5 = - \frac{\kappa\xi\veps}{6}$, $c_6 = 0$. 
In addition, we write $\textcolor{green}{\ddot{\Delta}}, \textcolor{red}{\ddot{X}_G}, \textcolor{blue}{\ddot{\Theta}}$ as $\dot{v}_z, \dot{v}_x, \dot{v}_\tta$, $\dot{\Dlt}, \dot{X}_G, \dot_{\Theta}$ as $v_z, v_x, v_\tta$, $\Dlt, X_G, \Theta$ as $r_z, r_x, r_\tta$.





\ssc*{Effective friction matrix}
\sss*{Mass matrix}
Consider the deterministic equation with mass according to the equations of motion mentioned above:
$$ m_\al \cdot \dot{v}_\al = \left[ F_{1\al}(\bbf{x}) + F_{2\ab}(\bbf{x}) \dot{v}_\bt \right] - m_\al \cdot \gamma_{\ab} v_\bt $$
we have 
$$ F_{1z} = - m_z a_6 = - m_z \tens F_{1x} = - m_x b_6 = 0 \tens F_{1\tta} = - m_\tta c_6 = 0 $$

$$ \agn{ &F_{2hzz} = - \frac{m_z a_5}{\Delta^{7/2}} \tens F_{2hzx} = 0 \tens F_{2hz\tta} = 0 \\
&F_{2hxz} = 0 \tens F_{2hxx} = - \frac{m_x b_5}{\Delta^{5/2}} \tens F_{2hx\tta} = - \frac{m_x b_4}{\Delta^{5/2}} \\
&F_{2h\tta z} = 0 \tens F_{2h\tta x} = - \frac{m_\tta c_4}{\Delta^{5/2}} \tens F_{2h\tta\tta} = - \frac{m_\tta c_5}{\Delta^{5/2}} }$$


Introduce the mass matrix as $M_{\alpha\beta} = \delta_{\alpha\beta} \cdot m_\alpha - F_{2h\alpha\beta}(\bbf{x})$:
$$ M = \left(
\begin{array}{ccc}
 m_z-\frac{15 \kappa  \xi  m_z}{8 \Delta ^{5/2}} & 0 & 0 \\
 0 & m_x -\frac{\kappa  \xi  \epsilon  m_x}{12 \Delta ^{5/2}} & \frac{\kappa  \xi  \epsilon  m_x}{12 \Delta ^{5/2}} \\
 0 & \frac{\kappa  \xi  \epsilon  m_{\theta }}{6 \Delta ^{5/2}} & m_{\theta }-\frac{\kappa  \xi  \epsilon  m_{\theta }}{6 \Delta ^{5/2}} 
\end{array} \right) $$
with its inverse matrix 
$$ M^{-1} = \left(
\begin{array}{ccc}
 \frac{1}{m_z-\frac{15 \kappa  \xi  m_z}{8 \Delta ^{5/2}}} & 0 & 0 \\
 0 & \frac{12 \Delta ^{5/2}-2 \kappa  \xi  \epsilon }{12 \Delta ^{5/2} m_x-3 \kappa  \xi  \epsilon  m_x} & \frac{\kappa  \xi  \epsilon }{3 m_{\theta } \left(\kappa  \xi  \epsilon -4 \Delta ^{5/2}\right)} \\
 0 & \frac{2 \kappa  \xi  \epsilon }{3 m_x \left(\kappa  \xi  \epsilon -4 \Delta ^{5/2}\right)} & \frac{12 \Delta ^{5/2}-\kappa  \xi  \epsilon }{12 \Delta ^{5/2} m_{\theta }-3 \kappa  \xi  \epsilon  m_{\theta }} \\
\end{array}
\right) $$
The inverse matrix at first-order approximation of $\kappa$ shows
$$ M^{-1} \approx \left(
\begin{array}{ccc}
 \frac{1}{m_z}+\frac{15 \kappa  \xi }{8 \Delta ^{5/2} m_z} & 0 & 0 \\
 0 & \frac{1}{m_x}+\frac{\kappa  \xi  \epsilon }{12 \Delta ^{5/2} m_x} & -\frac{\kappa \xi  \epsilon}{12 \Delta ^{5/2} m_{\theta }} \\
 0 & -\frac{\kappa \xi  \epsilon}{6 \Delta ^{5/2} m_x} & \frac{1}{m_{\theta }}+\frac{\kappa  \xi  \epsilon }{6 \Delta ^{5/2} m_{\theta }} 
\end{array}
\right) $$



\sss*{Fokker-Planck equation for friction matrix}
Let $\bb{P}(q,t)$ be the probability density function to find a particle in $[q,q+dq]$, as the general coordinate $q$ satisfies
$$ \dot{q}(t) = - W^\prime(q) + \xi (t) $$
where $\xi(t)$ refers to the Wiener process. We have the Fokker-Planck equation as
$$ \pder[\bb{P}(q,t)]{t} = \pder{q} \left[ \frac{dW(q)}{dq} + T \pder{q} \right] \bb{P}(q,t) $$


Suppose that $\bbf{x}, \bbf{v}$ refer to the position and velocity, respectively. We consider the following deterministic equation
$$ d \bbf{x} = \bbf{v} dt $$
$$ d \bbf{v} = - \bbf{U} dt - \bbf{\nabla} \phi(\mathbf{x}) dt $$
where $\phi(\bbf{x})$ is the external potential only including gravity. We assume that $\bbf{U}$ are generated by hydrodynamic interactions, which do not however affect the equilibrium Gibbs-Boltzmann distribution shown as
$$ P_{eq} (\mathbf{x},\mathbf{v}) = \frac{1}{\bar{Z}} \exp \left( - \frac{\beta \mathbf{v}^2}{2} - \beta \phi(\mathbf{x}) \right) $$
Note, $\pder[P]{x_\alpha} = P \left( -\beta \pder[\phi]{x_\alpha} \right)$, $\pder[P]{v_\alpha} = P \left( -\beta v_\alpha \right)$, and $\bt^{-1} = k_B T \overset{k_B = 1}{\longrightarrow} T$.


Exploit the Fokker-Planck equation on the distribution probability $P$ above as a function of time $t$. %$ \gamma_{\alpha\beta} = ?$
$$ \agn{ 
\pder[P]{t} &= \pder{v_\alpha} \left[ (U_\alpha + \nabla_\alpha \phi) P + T \pder[P]{v_\al} \right] + \pder{x_\alpha} \llp T \pder[P]{v_\al} \rrp \\ % rang 1
&= \pder{v_\alpha} \left[ (U_\alpha + \nabla_\alpha \phi) P + T \pder[P]{v_\beta} \pder[v_\bt]{v_\al} \right] + \pder{x_\alpha} \left[ T \cdot P (- \bt v_\al) \right] \\ % rang 2
&= \pder{v_\alpha} \left[ T \gamma_{\alpha\beta} \pder[P]{v_\beta} + U_\alpha P + \pder[\phi]{x_\alpha} P \right] - \pder{x_\alpha} \llp v_\alpha P \rrp % rang 3
}$$
%$$ \bgn
%\pder{x_\alpha} \left[ (U_\alpha + \nabla_\alpha \phi) P + T \gamma_{\alpha\beta} \pder[P]{v_\beta} \right] &= \pder{X_\alpha} \left[ \frac{dV}{dX_\alpha} P + T \pder{X_\alpha} P + T \pder{V_\alpha} P \right] \\
%&= \pder{X_\alpha} \left[ (\nabla_\alpha\phi)P + \cancel{T} \cdot P \left( -\cancel{\beta} \pder[\phi]{X_\alpha} \right) + T \pder{V_\alpha} P \right] = \pder{X_\alpha} \left[ T \pder{V_\alpha} P \right] \\
%&= \pder{X_\alpha} \left[ \cancel{T} \cdot P \left( -\cancel\beta V_\alpha \right) \right] = - \pder{X_\alpha} V_\alpha P
%end{align} $$
The last two terms would vanish since
$$ \pder{v_\alpha} \left( \pder[\phi]{x_\alpha} P \right) = \cancel{ \left( \pder{v_\alpha} \pder[\phi]{x_\alpha} \right) } \cdot P + \pder[\phi]{x_\alpha} \cdot \pder[P]{v_\alpha} = \pder[\phi]{x_\alpha} \cdot P (-\beta v_\alpha) $$
$$ \pder{x_\alpha} \llp v_\alpha P \rrp = \cancel{ \left( \pder[v_\alpha]{x_\alpha} \right) } P + v_\alpha \left( \pder[P]{x_\alpha} \right) = v_\alpha \cdot P \cdot \left( -\beta \pder[\phi]{x_\alpha} \right) $$
So we have
$$ \pder[P]{t} = \pder{v_\alpha} \llp T \gamma_{\alpha\beta} \pder[P]{v_\beta} + U_\alpha P \rrp = \pder{v_\alpha} \llp - \gamma_{\alpha\beta} v_\beta P + U_\alpha P \rrp $$
Therefore, at equilibrium $\pder[P]{t} = 0$, we obtain the GB distribution for the steady state if 
$$ U_\alpha = \gamma_{\alpha\beta} v_\beta $$


We have for small velocities that
$$ U_\al = \gamma_{\alpha\beta} v_\beta = \lambda_{\alpha\beta}(\mathbf{x}) v_\beta + \Lambda_{\alpha\beta\gamma} (\mathbf{x}) v_\beta v_\gamma $$ 
where the term $\lambda_{\alpha\beta}(\mathbf{x})$ is just the friction tensor without any elastic effects. Additional efforts should be taken on the second term by symmetry. We would like to have 
$$ \gamma_{\alpha\beta} = \lambda_{\alpha\beta} + \gamma_{2\alpha\beta} \tens \gamma_{2\alpha\beta} = \Gamma_{\alpha\beta\gamma} v_\gamma $$
Consequently, we have 
$$ \Gamma_{\alpha\beta\gamma} (\mathbf{x}) v_\beta v_\gamma = \Lambda_{\alpha\beta\gamma} (\mathbf{x}) v_\beta v_\gamma $$
Without loss of generality, we take 
$$\Lambda_{\alpha\beta\gamma} = \Lambda_{\alpha\gamma\beta}$$
which then gives 
$$\Gamma_{\alpha\beta\gamma} + \Gamma_{\alpha\gamma\beta} = 2 \Lambda_{\alpha\beta\gamma}$$ 
In fact, velocity terms on different directions contribute equally for products, so 
$$\Lambda_{\alpha\beta\gamma} = \Lambda_{\alpha\gamma\beta}$$
Also, mutual interactions means that terms with $v_\al$ contribute equally toward $\gma_{\ab} v_\bt$, hence we obtain another constraint 
$$\Gamma_{\alpha\beta\gamma} = \Gamma_{\beta\alpha\gamma}$$


Following the format of Langevin equation, $\gamma_{\ab}$ matrix above only contains terms about first derivatives %does not contain coefficients $(a/b/c)_{5/6}$
$$ \agn{
U_z = \gamma_{z\beta} v_\beta &= a_1 \frac{\dot{\Delta}}{\Delta^{3/2}} + a_2 \frac{\dot{\Delta}^2}{\Delta^{9/2}} + a_3 \frac{\dot\Theta^2 + \dot{X}^2}{\Delta^{7/2}} + a_4 \frac{\dot\Theta \dot{X}}{\Delta^{7/2}}\\
U_x = \gamma_{x\beta} v_\beta &= b_1 \frac{\dot{X}}{\sqrt\Delta} + b_2  \frac{\dot\Delta \dot{X}}{\Delta^{7/2}} + b_3 \frac{\dot\Delta \dot\Theta}{\Delta^{7/2}} \\
U_\tta = \gamma_{\tta\beta} v_\beta &= c_1 \frac{\dot\Theta}{\sqrt\Delta} + c_2 \frac{\dot\Delta \dot\Theta}{\Delta^{7/2}} + c_3 \frac{\dot\Delta \dot{X}}{\Delta^{7/2}} 
} $$
%with reduced parameters like $a_1 = \xi$, $b_1 = \frac{2 \veps \xi}{3}$, and so on for convenience. %\{Based on the equilibrium hypothesis from Fokker-Planck equation  (\textit{need details?}) ...\} 
From this we take the first equation for instance, finding 
$$ \sum_\al \lambda_{z\al} v_\al = \xi \frac{v_z}{\Dlt^{3/2}} \tens \Rightarrow \tens \lambda_{zz} = \frac{\xi}{\Dlt^{3/2}} \fives \lambda_{zx} = 0 \fives \lambda_{z\tta} = 0 $$
Similarly, we have
$$ \sum_\al \lambda_{x\al} v_\al = \frac{2 \eps \xi v_x}{3 \Dlt^{1/2}} \tens \Rightarrow \tens \lambda_{xz} = 0 \fives \lambda_{xx} = \frac{2 \eps \xi}{3 \Dlt^{1/2}} \fives \lambda_{x\tta} = 0 $$
$$ \sum_\al \lambda_{\tta\al} v_\al = \frac{4 \eps \xi v_\tta}{3 \Dlt^{1/2}} \tens \Rightarrow \tens \lambda_{\tta z} = 0 \fives \lambda_{\tta x} = 0 \fives \lambda_{\tta\tta} = \frac{4 \eps \xi}{3 \Dlt^{1/2}} $$


Consider 
$$ \sum_{\ab} \Lambda_{z\ab} v_\al v_\bt = \frac{21 \kxi v_z^2}{4 \Dlt^{9/2}} - \frac{\kxi \llp v_x^2 + v_\tta^2 \rrp}{4 \Dlt^{7/2}} + \frac{\kxi v_x v_\tta}{2 \Dlt^{7/2}} $$
which furnishes
$$ \Gamma_{zzz} = \frac{21\kxi}{4 \Dlt^{9/2}} \tens \Gamma_{zxx} = - \frac{\kxi}{4 \Dlt^{7/2}} \tens \Gamma_{z\tta\tta} = - \frac{\kxi}{4 \Dlt^{7/2}} $$
Again
$$ \sum_{\ab} \Lambda_{x\ab} v_\al v_\bt = \frac{19 \kxi \eps v_z v_x}{24 \Dlt^{7/2}} - \frac{\kxi \eps v_x v_\tta}{6 \Dlt^{7/2}}  $$
we get
$$ \Gamma_{xxz} + \Gamma_{xzx} = \frac{19 \kxi \eps}{24 \Dlt^{7/2}} $$
The symmetry $\Gamma_{\ab\gma} = \Gamma_{\bt\al\gma}$ now gives
$$ \Gamma_{xxz} = \frac{19 \kxi \eps}{24 \Dlt^{7/2}} - \Gamma_{xzx} = \frac{19 \kxi \eps}{24 \Dlt^{7/2}} - \Gamma_{zxx} = \frac{\kxi}{\Dlt^{7/2}} \llp \frac{19 \eps}{24} + \frac{1}{4} \rrp $$


Therefore, we could resolve all coefficients $\lambda_{\ab}$ and $\Gamma_{\ab\gma}$ by this way:

$$ \lambda_{\ab} = \left(\begin{array}{ccc}\frac{a_1}{\Delta ^{3/2}} & 0 & 0 \\0 & \frac{b_1}{\sqrt{\Delta }} & 0 \\0 & 0 & \frac{c_1}{\sqrt{\Delta }}\end{array}\right) $$

$$ \Gamma_{z\ab} = \left(\begin{array}{ccc}\frac{a_2}{\Delta ^{9/2}} & 0 & 0 \\0 & \frac{a_3}{\Delta ^{7/2}} & \frac{a_4 + b_3 - c_3}{2 \Delta ^{7/2}} \\0 & \frac{a_4 - b_3 + c_3}{2 \Delta ^{7/2}} & \frac{a_3}{\Delta ^{7/2}}\end{array}\right) \fives 
\Gamma_{x\ab} = \left(\begin{array}{ccc}0 & \frac{a_3}{\Dlt^{7/2}} & \frac{a_4+b_3-c_3}{2 \Delta ^{7/2}} \\\frac{b_2-a_3}{\Delta ^{7/2}} & 0 & 0 \\\frac{-a_4+b_3+c_3}{2 \Delta ^{7/2}} & 0 & 0\end{array}\right) \fives 
\Gamma_{\tta\ab} = \left(\begin{array}{ccc}0 & \frac{a_4-b_3+c_3}{2 \Delta ^{7/2}} & \frac{a_3}{\Dlt^{7/2}} \\\frac{-a_4+b_3+c_3}{2 \Delta ^{7/2}} & 0 & 0 \\\frac{c_2-a_3}{\Delta ^{7/2}} & 0 & 0\end{array}\right) $$


Combine those coefficients together by $\gma_{\ab} = \lambda_{\ab} + \Gamma_{\ab\gma} v_\gma$, we obtain
$$ \gma_{\ab} = \left(
\begin{array}{ccc}
 \frac{a_1}{\Delta ^{3/2}}+\frac{a_2 v_z}{\Delta ^{9/2}} & \frac{v_{\theta } \left(a_4+b_3-c_3\right)}{2 \Delta ^{7/2}}+\frac{a_3 v_x}{\Delta ^{7/2}} & \frac{v_x \left(a_4-b_3+c_3\right)}{2 \Delta ^{7/2}}+\frac{a_3 v_{\theta }}{\Delta ^{7/2}} \\
 \frac{v_{\theta } \left(a_4+b_3-c_3\right)}{2 \Delta ^{7/2}}+\frac{a_3 v_x}{\Delta ^{7/2}} & \frac{\left(b_2-a_3\right) v_z}{\Delta ^{7/2}}+\frac{b_1}{\sqrt{\Delta }} & \frac{v_z \left(-a_4+b_3+c_3\right)}{2 \Delta ^{7/2}} \\
 \frac{v_x \left(a_4-b_3+c_3\right)}{2 \Delta ^{7/2}}+\frac{a_3 v_{\theta }}{\Delta ^{7/2}} & \frac{v_z \left(-a_4+b_3+c_3\right)}{2 \Delta ^{7/2}} & \frac{\left(c_2-a_3\right) v_z}{\Delta ^{7/2}}+\frac{c_1}{\sqrt{\Delta }} \\
\end{array}
\right)$$
and its 1-order approximation of $\kpa$:
$$ \gma_{\ab} \approx \left(
\begin{array}{ccc}
 \frac{\xi }{\Delta ^{3/2}}+\frac{21 \kappa  \xi  v_z}{4 \Delta ^{9/2}} & \frac{\kappa  \xi  \left((\epsilon +3) v_{\theta }-3 v_x\right)}{12 \Delta ^{7/2}} & -\frac{\kappa  \left(\xi  \left(3 v_{\theta }+(\epsilon -3) v_x\right)\right)}{12 \Delta ^{7/2}} \\
 \frac{\kappa  \xi  \left((\epsilon +3) v_{\theta }-3 v_x\right)}{12 \Delta ^{7/2}} & \frac{2 \xi  \epsilon }{3 \sqrt{\Delta }}+\frac{\kappa  \xi  (19 \epsilon +6) v_z}{24 \Delta ^{7/2}} & -\frac{\kappa  \left(\xi  (\epsilon +1) v_z\right)}{4 \Delta ^{7/2}} \\
 -\frac{\kappa  \left(\xi  \left(3 v_{\theta }+(\epsilon -3) v_x\right)\right)}{12 \Delta ^{7/2}} & -\frac{\kappa  \left(\xi  (\epsilon +1) v_z\right)}{4 \Delta ^{7/2}} & \frac{4 \xi  \epsilon }{3 \sqrt{\Delta }}+\frac{\kappa  \xi  (19 \epsilon +3) v_z}{12 \Delta ^{7/2}} \\
\end{array}
\right) $$




\sss*{Effective friction matrix}
Therefore, the deterministic equation turns to 
$$ m_{\al} \cdot \dot{v}_\al &- F_{2\ab}(\bbf{x}) \dot{v}_\bt = M_{\ab} \dot{v}_\beta = F_{1\al}(\bbf{x}) - m_\al \cdot \gamma_{\ab} v_\beta $$
and then we could finally find the effective friction matrix as $\gamma_\mrm{eff} =  M_{\ab}^{-1} m_\al \cdot \gma_{\ab} v_\bt$ 
$$ \dot{v}_\bt &= M_{\ab}^{-1} \left( F_{1\al}(\bbf{x}) - m_\al \cdot \gamma_{\ab} v_\beta \right) $$
 $$ \gamma_\mrm{eff} = M_{\ab}^{-1} \cdot \left(\begin{array}{ccc}m_Z & 0 & 0 \\0 & m_X & 0 \\0 & 0 & m_\Theta \end{array}\right) \cdot \gamma_{\ab} $$
with elements below:
$$ \agn{
\gamma_{\mrm{eff},zz} &\approx \frac{\xi }{\Delta ^{3/2}} +\kappa  \left(\frac{15 \xi ^2}{8 \Delta^4}+\frac{21 \xi v_z}{4 \Delta ^{9/2}}\right) \\ % rang 1
\gamma_{\mrm{eff},xx} &\approx \frac{2 \xi  \epsilon }{3 \sqrt{\Delta }}+\frac{\kappa  \xi  \left(4 \sqrt{\Delta } \xi  \epsilon^2 + 18 v_z + 57 \epsilon  v_z \right)}{72 \Delta ^{7/2}} \\ % rang 2
\gamma_{\mrm{eff},\theta\theta} &\approx \frac{4 \xi  \epsilon }{3 \sqrt{\Delta }}+\frac{\kappa  \xi  \left(8 \sqrt{\Delta } \xi  \epsilon ^2+57 \epsilon  v_z+9 v_z\right)}{36 \Delta ^{7/2}} \\ % rang 3
\gamma_{\mrm{eff},xz} &= \gamma_{\mrm{eff},zx} \approx \frac{\kappa  \xi  \left((\epsilon +3) v_{\theta }-3 v_x \right)}{12 \Delta ^{7/2}}  \\ % rang 4
\gamma_{\mrm{eff},\theta z} &= \gamma_{\mrm{eff},z\theta} \approx \frac{\kappa \xi  \left( (3 -\epsilon) v_x - 3 v_{\theta } \right)}{12 \Delta ^{7/2}} \\ % rang 5
\gamma_{\mrm{eff},\theta x} &= \gamma_{\mrm{eff},x\theta} \approx -\frac{\kappa \xi  \left(16 \Delta ^3 \xi  \epsilon ^2+36 \Delta ^{5/2} (\epsilon +1) v_z\right)}{144 \Delta ^6} % rang 6
} $$






\ssc*{Modified noise correlator amplitude} %\label{corr}
After the effective friction matrix $\gamma_\rm{eff}$, we consider the random forces and their correlator amplitudes. For the 1D case in the bulk, we only need the square root of friction coefficient. Similarly, we could suppose that $\gamma_{\mrm{eff}} \approx \Psi + \kappa \Phi$, %where $\Psi$ is zero-order matrix of $\kpa $, while $\Phi $ the first-order one.
%$\Psi_i = \mrm{SeriesCoefficients} [\mrm{Series}[\gma_{\mrm{eff},ii},\{\kpa,0,0\} ], 0]$ \\
%$\Phi_i = \mrm{SeriesCoefficients} [\mrm{Series}[\gma_{\mrm{eff},ij},\{\kpa,0,1\} ], 1]$ \\
as well as $\gamma^{1/2}_\mrm{eff} \approx \psi + \kpa\chi$, then we have
$$ \gamma_{\mrm{eff}} = \gamma_{\mrm{eff}}^{1/2}\gamma_{\mrm{eff}}^{1/2} = (\psi + \kappa\chi)(\psi + \kappa\chi) \approx \psi\psi + \kappa (\psi\chi + \chi\psi) $$
so we resolve \(\psi_{ij} =\sqrt{\Psi_{ij} }\), and $\chi_{ij} = \frac{\Phi_{ij}}{\sqrt{\Psi_{ii}} + \sqrt{\Psi_{jj}}}$. 
%Consider the case without unit mass, we have to calculate the inverse mass matrix as an analogy of $\frac{1}{m}$, hences $\gamma_{\mrm{eff}}^{1/2} \to \llp \gamma_{\mrm{eff}} \cdot M^{-1} \rrp^{1/2}$ as an asymmetric matrix. Even though we could neglect the non-diagonal elements as the cross-correlated noise since $\kpa \ll 1$, we are still motivated to clarify these terms by proper treatment, such as the diagonalisation. However, we could hardly furnish a simple diagonalized matrix, numerical method would be expected. After extracting noise eigenvalues, we exploit the inverse base transform to furnish the exact contribution on each direction. Further discussion see subsection \ref{Discretisation algorithm}.


The results seem plausible and enough with the first-order correction of $\kappa$. However, as for $\gamma_\rm{eff}$, several velocities have been included. In 1D case, we make Laplace and the its inverse transform for solutions, while here we have to consider that as a matrix equation
%$$ \dot{\bbf{v}}_{3\times1} = - \gamma_{\rm{eff},3\times3} \cdot \bbf{v}_{3\times1} + M^{-1}_{3\times3} \cdot \delta F_{3\times1} $$
%and consider the corresponding Laplace transform
$$ \widetilde{\dot{\bbf{v}}} = - \widetilde{\gamma_{\rm{eff}} \cdot \bbf{v}} + \widetilde{M^{-1} \cdot \bbf{\delta F}} $$
Note, $\gamma_{\mrm{eff}}$ is not a constant matrix, which should be included inside the Laplace transform.
%We could not extract this friction matrix outside the Laplace transform, while similar for $M^{-1}$. 
%We try the Laplace transform directly on $-\widetilde{\gamma_{\rm{eff}} \cdot \bbf{v}}$, obtaining

%The Laplace transform of multiplication would be quite complex, since 
%$$ \widetilde{f(t)g(t)}(p) = \frac{1}{2\pi i} \lim_{T \to \infty} \int_{c-iT}^{c+iT} \widetilde{f}(s) \widetilde{g}(p-s) ds $$ 
%The integration is done along the vertical line $\Re(s)=c$ that lies entirely within the region of convergence of $\widetilde{f}$. For example, $f(t) = e^t$ does not possess a convergent Laplace integral if $\Re p>1$ or if $\Re p<1$. The strip of convergence has contracted to a line: the integral converges only where $\Re p=1$, and even then not exactly at $p=1$. We'd like to consider this part after getting the proper formula of $v_i$. For instance, if $f(t) = e^{-\gamma t}$ with $t>0$, we have the convergence strip if $\Re p > - \gamma$, which could be satisfied in our issue if we have a constant $\gamma$.


Dissect $\gamma_\rm{eff}$ as $\gamma_{\rm{eff}} = \gamma_{0} + \gamma_{1}(\kpa) + \gamma_{1v}(\kpa,v_i)$, 
$$ \gamma_0 = \left(\begin{array}{ccc} \frac{\xi }{\Delta ^{3/2}} & 0 & 0 \\0 & \frac{2 \xi  \epsilon }{3 \sqrt{\Delta }} & 0 \\0 & 0 & \frac{4\xi \eps}{3\sqrt{\Delta}}  \end{array}\right) \tens \gamma_1 = \left(\begin{array}{ccc} \frac{15 \kappa \xi ^2}{8 \Delta ^4} & 0 & 0 \\0 &  \frac{ \kappa  \xi^2  \epsilon ^2}{18 \Delta^3} & -\frac{\kappa \xi^2 \eps^2}{9 \Delta^3} \\0 & -\frac{\kappa \xi^2 \eps^2}{9 \Delta^3} &  \frac{2 \kappa \xi^2 \eps^2}{9 \Delta^3} \end{array}\right) $$
$$ \gamma_{1v} = \left(\begin{array}{ccc} \frac{21 \kappa\xi v_z}{4 \Delta ^{9/2}} & \frac{\kxi  \left((\epsilon +3) v_{\theta }-3 v_x\right)}{12 \Delta ^{7/2}} & \frac{\kxi  \left( (3 - \epsilon) v_x - 3 v_{\theta} \right)}{12 \Delta ^{7/2}} \\ \frac{\kxi  \left((\epsilon +3) v_{\theta }-3 v_x\right)}{12 \Delta ^{7/2}} & \frac{\kappa\xi (6 + 19 \eps) v_z}{24 \Delta^{7/2}} & - \frac{\kappa \xi (\eps+1) v_z}{4 \sqrt{\Delta}} \\ \frac{\kxi  \left( (3 - \epsilon) v_x - 3 v_{\theta} \right)}{12 \Delta ^{7/2}} & - \frac{\kappa \xi (\eps+1) v_z}{4 \sqrt{\Delta}} & \frac{\kappa\xi (19 \eps + 3) v_z}{12 \Delta^{7/2}} \end{array}\right) $$
where $\gamma_0$ is constant matrix, independent on $\kpa$; $\gma_1$ depends on $\kpa$; and $\gamma_{1v}$ depends on $\kpa$ and velocities $v_i$. 
Hence we could separate the transform as $\widetilde{\gamma_{\rm{eff}} \cdot \bbf{v}} = \gamma_0 \cdot \widetilde{\bbf{v}} + \gamma_1 \cdot \widetilde{\bbf{v}} + \widetilde{\gamma_{1v} \cdot \bbf{v}}$. Since $\gma_0$ is a diagonal matrix, we write $\gma_{i0} = \gma_{i0}$ for the convenience. Also, we suppose that $\gamma_{1v,ij} = g_{ij\alpha} v_\alpha$, where $g_{ij\alpha}$ refers to the coefficient of $v_\alpha$ in $\gamma_{1v,ij}$, such as $g_{12x} = -\frac{\kxi v_x}{4 \Delta^{7/2}}$. A symmetric $\gamma_{\rm{eff}}$ results in symmetric $\gamma_0$ and $\gamma_1$, so is $g_{ij\alpha}$. 

It would be raisonnable to consider the perturbation on $\kpa$, for this elastic compliance parameter $\kappa \ll 1$ (about $10^{-4} \sim 10^{-3}$). We write $\bbf{v} = \bbf{v}_0 + \bbf{v}_1$, where the former is on 0 order while the latter 1 order. Similarly, the mass matrix and random forces would be treated in the same way.
$$ \dot{\bbf{v}} &= \dot{\bbf{v}}_0 + \dot{\bbf{v}}_1 = - \gma_{\mrm{eff}} \cdot \bbf{v} + M^{-1} \cdot \bbf{\dlt F} = - (\gma_0 + \gma_1 + \gma_{1v}) \cdot (\bbf{v}_0 + \bbf{v}_1) + (M^{-1}_0 + M^{-1}_1) \cdot (\bbf{\dlt F}_0 + \bbf{\dlt F}_1 ) $$
We only keep terms of 0 and 1 order of $\kpa$:
$$ \dot{\bbf{v}}_0 = - \gma_0 \cdot \bbf{v}_0 + M^{-1}_0 \cdot \bbf{\dlt F}_0 $$
$$ \dot{\bbf{v}}_1 = - \gma_0 \cdot \bbf{v}_1 - \gma_1 \cdot \bbf{v}_0 - \gma_{1v} \cdot \bbf{v}_0 + M^{-1}_0 \cdot \bbf{\dlt F}_1 + M^{-1}_1 \cdot \bbf{\dlt F}_0 $$
After Laplace transform, we have
$$ s \widetilde{\dot{\bbf{v}}}_0 - \bbf{v}(0) = - \gma_0 \cdot \wtd{\bbf{v}}_0 + M^{-1}_0 \cdot \wtd{\bbf{\dlt F}_0} $$
$$ s \widetilde{\dot{\bbf{v}}}_1 = - \gma_0 \cdot \wtd{\bbf{v}}_1 - \gma_1 \cdot \wtd{\bbf{v}}_0 - \wtd{\gma_{1v} \cdot \bbf{v}_0} + M^{-1}_0 \cdot \wtd{\bbf{\dlt F}_1} + M^{-1}_1 \cdot \wtd{\bbf{\dlt F}_0} $$
Note $\mathcal{L}_t\left[\int_0^t f(\tau ) g(t-\tau ) \, d\tau \right](s) = \left(\mathcal{L}_t[f(t)](s)\right) \left(\mathcal{L}_t[g(t)](s)\right)$. 0-order solutions are rather simple:
$$ v_{i0}(t) = v_{i0}(0) e^{-\gma_{i0} t} + \int_0^t \mrm{d}\tau \frac{\delta F_{i0} (\tau)}{m_i} \exp\left[-\gma_{i0}(t-\tau)\right] $$
Follow the same process we have done previously, we get the amplitude of noise correlator:
$$ \llang \delta F_{i0} (\tau_1) \delta F_{j0} (\tau_2) \rrang = 2 k_B T m_i \gamma_{i0} \delta_{ij} \delta(\tau_1 - \tau_2) $$

As for the 1-order correction $v_{i1}$, we have
$$ (s + \gma_{i0}) \wtd{v_{i1}} &= - \sum_j \gma_{1,ij} \wtd{v_{i0}} - \sum_j \sum_k g_{ijk}  (\wtd{v_{j0} \cdot v_{k0}}) + M^{-1}_{0i} \wtd{\dlt F_{i1}} + \sum_j M^{-1}_{1,ij} \wtd{\dlt F_{j0}} $$
Laplace and its inverse transform have been calculated. To be clear, we decompose $\bbf{v}_1$ as
$$ \bbf{v}_{1} = \bbf{v}_{gv} + \llp \bbf{v}_{vv} + \bbf{v}_{vf} + \bbf{v}_{fv} + \bbf{v}_{ff} \rrp + \bbf{v}_{fm} + \bbf{v}_{mf} $$
with the following expressions along the direction $i$:

$$  v_{i,gv} = \frac{\gma_{1,ij}}{\gma_{i0} - \gma_{j0}} \lls \llp e^{- \gma_{i0} t} - e^{-\gma_{j0} t} \rrp v_j (0) + \int_0^t \mrm{d}\tau \frac{\dlt F_{j0}(\tau)}{m_j} \lls e^{- \gma_{i0} (t-\tau)} - e^{-\gma_{j0} (t-\tau)} \rrs \rrs $$
$$ v_{i,vv} = - g_{ijk} v_j (0) v_k (0) \cdot \frac{e^{-(\gma_{j0} + \gma_{k0}) t} - e^{-\gma_{i0} t}}{\gma_{i0} - \gma_{j0} - \gma_{k0}} $$
$$ v_{i,fv} = - g_{ijk} v_k (0) \int_0^t \mrm{d}\tau \dlt F_{j0} (\tau) \frac{e^{-(\gma_{j0} + \gma_{k0}) (t-\tau)} - e^{-\gma_{i0} (t-\tau)}}{m_j \llp \gma_{i0} - \gma_{j0} - \gma_{k0} \rrp} $$
$$ v_{i,vf} = - g_{ijk} v_j (0) \int_0^t \mrm{d}\tau \dlt F_{k0} (\tau) \frac{e^{-(\gma_{j0} + \gma_{k0}) (t-\tau)} - e^{-\gma_{i0} (t-\tau)}}{m_k \llp \gma_{i0} - \gma_{j0} - \gma_{k0} \rrp} $$
$$ v_{i,ff} = - \frac{g_{ijk}}{m_j m_k} \int_0^t \mrm{d}\tau  \int_0^\tau \mrm{d}x \dlt F_{j0} (x) e^{-\gma_{j0} (\tau - x)}  \int_0^\tau \mrm{d}y \dlt F_{k0} (y) e^{-\gma_{k0} (\tau - y)} e^{- \gma_{i0} (t-\tau)} $$ 
$$ v_{i,fm} = \int_0^t \mrm{d}\tau \frac{\dlt F_{i1} (\tau)}{m_i} e^{-\gma_{i0} (t-\tau)} \tens
 v_{i,mf} = M^{-1}_{1,ij} \int_0^t \mrm{d}\tau \dlt F_{j0} (\tau) e^{-\gma_{i0} (t-\tau)} $$

%Generally, we ignore the correlation between velocities and random forces. 
However, higher order correlation functions would be introduced due to $v_{vv}, v_{fv}, v_{vf}, v_{ff}$ while calculating noise correlator amplitudes and diffusion coefficients, such as $\lla v_i v_j v_k \rra$, $\lla v_i v_j \dlt F_{k0} \rra$, $\lla v_i \dlt F_{j0} \dlt F_{k0} \rra$, $\lla \dlt F_{i0} \dlt F_{j0} \dlt F_{k0} \rra$. In fact, we pose that there is no correlation between velocities and random forces $\lla v_i \dlt F_j \rra = 0$, as well as $\lla \dlt F_{i0} \rra = 0$. Therefore, we are inclined to neglect these odd-power terms below.

Note, as for $\bbf{v}_{gv}$, %if $\gma_{i0} = \gma_{j0}$,
$ \lim_{\gma_{i0} \to \gma_{j0}} \frac{e^{- \gma_{i0} t} - e^{-\gma_{j0} t}}{\gma_{i0} - \gma_{j0}} = - t e^{-\gma_{i0} t} $. With all coefficients known, we could resolve $\bbf{v}_1$. Then we take $v_{z1}(t)$ for instance for the following calculation.

$$ v_{z1}(t) = - v_z(0) \gma_{1,zz} t e^{-\gma_{z0}t} + \int_0^t \mrm{d}\tau e^{-\gma_{z0} (t-\tau)} \llc \lls \frac{\delta F_{z1}(\tau)}{m_z} + M^{-1}_{zz1} \delta F_{z0}(\tau) \rrs - \gma_{1,zz} (t-\tau) \frac{\delta F_{z0}(\tau)}{m_z} \rrc $$

Still, we consider the velocity square average up to 1-order $\kpa$:
$$ \lla v_z^2(t) \rra = \lla \lls v_{z0}(t) + v_{z1}(t) \rrs^2 \rra \approx \lla v_{z0}^2(t) \rra + 2 \lla v_{z0}(t) v_{z1}(t) \rra $$

Suppose there exists the correlation between 0-order and 1-order random force, $\lla \dlt F_{z0}(\tau_1) \dlt F_{z1}(\tau_2) \rra = K_z \cdot \delta(\tau_1 - \tau_2)$. So at long time limit $t \to \infty$, $\lla v_z^2(t) \rra$ would converge to
$$ \lla v_z^2(t) \rra = k_B T \lls \frac{1}{m_z} + 2 \llp M^{-1}_{1,zz} - \frac{\gma_{1,zz}}{2 m_z \gma_{z0}} \rrp \rrs + \frac{K}{m_z^2 \gma_{z0}} $$ 
Since $\lla v_z^2(t) \rra = \frac{k_B T}{m_z}$, we obtain the amplitude $K_z$
$$ K_z &= k_B T m_z \llp \gma_{1,zz} - 2\gma_{z0} m_z M^{-1}_{1,zz} \rrp $$
Hence the modified noise amplitude of $z$ up to 1-order correction turns to
$$ \agn{
& \lla \dlt F_z(\tau_1) \dlt F_z(\tau_2) \rra \approx \lla \dlt F_{z0}(\tau_1) \dlt F_{z0}(\tau_2) \rra + 2 \lla \dlt F_{z0}(\tau_1) \dlt F_{z1}(\tau_2) \rra \\ % rang 2
&= 2 k_B T m_z \gma_{z0} \dlt(\tau_1 - \tau_2) + 2 k_B T m_z \llp \gma_{1,zz} - 2\gma_{z0} m_z M^{-1}_{1,zz} \rrp \dlt(\tau_1 - \tau_2) \\ % rang 3
&= 2 k_B T m_z \dlt(\tau_1 - \tau_2) \cdot \llp \gma_{z0} + \gma_{1,zz} - 2\gma_{z0} m_z M^{-1}_{1,zz} \rrp % rang 4
} $$
Note $M^{-1}_{1,zz} = \frac{15\kxi}{8 \Delta^{5/2} m_z}$, $\gma_{z0} + \gma_{1,zz} = \frac{\xi}{\Delta^{3/2}} + \frac{15 \kpa \xi^2}{8 \Delta^4}$, we calculate
$$ \gma_{z0} + \gma_{1,zz} - 2\gma_{z0} m_z M^{-1}_{1,zz} = \frac{\xi}{\Delta^{3/2}} - \frac{15 \kpa \xi^2}{8 \Delta^4} $$
and then an amazingly concise result:
$$ \lla \dlt F_z(\tau_1) \dlt F_z(\tau_2) \rra = 2 k_B T m_z \dlt(\tau_1 - \tau_2) \cdot \llp \gma_{z0} - \gma_{1,zz} \rrp $$
which is always valid at 1-order correction. Since $\gma_{z0} = \frac{a_1}{\Delta^{3/2}}$, $\gma_{1,zz} = - \frac{a_1 a_5}{\Delta^4}$, $M^{-1}_{1,zz} = - \frac{a_5}{\Delta^{5/2} m_z}$, we verify
$$ \gma_{z0} + \gma_{1,zz} - 2\gma_{z0} m_z M^{-1}_{1,zz} = \frac{a_1}{\Delta ^{3/2}}+\frac{a_5 a_1}{\Delta ^4} = \gamma _{z0}-\gamma _{1,zz} $$

Furthermore, we could repeat the same procedure for $v_{1x}$ and $v_{1\tta}$, deriving the modified noise correlator amplitudes $K_x$ and $K_\tta$. There are non-zero non-diagonal elements in $\gma_1$, so we get additional terms shown below:
%Further treatment would be needed for other components due to non-zero non-diagonal elements $\gamma_{0x\theta} = \gamma_{0\theta x} \neq 0$. 


$$ \agn{
v_{x1}(t) &= - v_x(0) \gamma _{1,xx} t e^{- \gamma _{x0} t} + \frac{v_{\tta}(0) \gamma_{1,x\tta} }{\gamma _{x0}-\gamma _{\tta0}} \llp e^{- \gamma _{x0} t} - e^{- \gamma _{\tta0} t} \rrp \\ % rang 1-1
& \fives - \gamma_{1,xx} \int_0^t \mrm{d} \tau (t-\tau) e^{-\gamma_{x0}(t-\tau)} \frac{\delta F_{x0}(\tau)}{m_x} \\ % rang 1-2
& \fives + \frac{\gamma_{1,x\theta}}{\gamma_{\theta0}-\gamma_{x0}} \int_0^t \mrm{d} \tau \llp e^{-\gamma_{\theta0}(t-\tau)} - e^{-\gamma_{x0}(t-\tau)} \rrp \frac{\delta F_{\theta0}(\tau)}{m_\tta} \\ % rang 1-3
& \fives + \int_0^t \mrm{d}\tau e^{-\gamma_{x0}(t-\tau)} \lls M^{-1}_{1,xx} \delta F_{x0}(\tau) + M^{-1}_{1,x\theta} \delta F_{\theta0}(\tau) + \frac{\dlt F_{x1}(\tau)}{m_x} \rrs % rang 1-4
} $$
$$ \agn{
v_{\tta1}(t) &= - v_{\theta}(0) \gamma _{1,\tta\tta} t e^{- \gamma _{\tta0} t} + \frac{v_x(0) \gamma _{1,\tta x}}{\gamma _{x0}-\gamma _{\tta0}} \llp e^{- \gamma _{x0} t} - e^{- \gamma _{\tta0} t} \rrp \\ % rang 2-1
& \fives + \frac{\gamma_{1,\tta x}}{\gamma_{\tta0} - \gamma_{x0}} \int_0^t \mrm{d}\tau  \llp e^{-\gamma_{\tta0}(t-\tau)} - e^{-\gamma_{x0}(t-\tau)} \rrp \frac{\delta F_{x0}(\tau)}{m_x} \\ % rang 2-2
& \fives - \gamma_{1,\tta\tta} \int_0^t \mrm{d}\tau (t-\tau) e^{-\gamma_{\tta0}(t-\tau)} \frac{\delta F_{\tta0}(\tau)}{m_\tta} \\ % rang 2-3
& \fives + \int_0^t \mrm{d}\tau e^{-\gamma_{\tta0}(t-\tau)} \lls M^{-1}_{1,\tta x} \delta F_{x0}(\tau) + M^{-1}_{1,\tta\tta} \delta F_{\theta0}(\tau) + \frac{\dlt F_{\tta1}(\tau)}{m_\tta}  \rrs % rang 2-4
} $$


Again, we suppose $\lla \dlt F_{x0}(\tau_1) \dlt F_{x1}(\tau_2) \rra = K_x \cdot \dlt(\tau_1 - \tau_2)$, and $\lla \dlt F_{\tta0}(\tau_1) \dlt F_{\tta1}(\tau_2) \rra = K_\tta \cdot \dlt(\tau_1 - \tau_2)$ for $\lla v_x^2 \rra$ and $\lla v_\tta^2 \rra$. At long time limit $t \to \infty$, they converge to:
$$ \agn{ \lla v_x^2(t) \rra &= k_B T \lls \frac{1}{m_x} + 2 \llp M^{-1}_{1,xx} - \frac{\gma_{1,xx}}{2 m_x \gma_{x0}} \rrp \rrs + \frac{K}{m_x^2 \gma_{x0}} \\
\lla v_\tta^2(t) \rra &= k_B T \lls \frac{1}{m_\tta} + 2 \llp M^{-1}_{1,\tta\tta} - \frac{\gma_{1,\tta\tta}}{2 m_\tta \gma_{x0}} \rrp \rrs + \frac{K}{m_\tta^2 \gma_{\tta0}} }$$ 
Since they should be equal to $\frac{k_B T}{m_x}, \frac{k_B T}{m_\tta}$, respectively, we get:
$$ \agn{
K_x &= k_B T m_x \llp \gma_{1,xx} - 2 m_x M^{-1}_{1,xx} \gma_{x0} \rrp \\ % rang 1
K_\tta &= k_B T m_\tta \llp \gma_{1,\tta\tta} - 2 m_\tta M^{-1}_{1,\tta\tta} \gma_{\tta0} \rrp % rang 2
} $$
Similar to the modified noise correlator on $z$, we obtain again concise results:
$$ \agn{ \lla \dlt F_x(\tau_1) \dlt F_x(\tau_2) \rra &= 2 k_B T m_x \dlt(\tau_1 - \tau_2) \cdot \llp \gma_{x0} - \gma_{1,xx} \rrp \\ % rang 1
\lla \dlt F_\tta(\tau_1) \dlt F_\tta(\tau_2) \rra &= 2 k_B T m_\tta \dlt(\tau_1 - \tau_2) \cdot \llp \gma_{\tta0} - \gma_{1,\tta\tta} \rrp } $$



\ssc*{Mean square displacement}
We have already obtained noise correlator amplitudes by $\lla v^2(t) \rra$. At the same time, we could also derive the mean square displacement (MSD) by $\lla v(0) v(t) \rra$. Reminder, there is no correlation between $v_i(t)$ and $\delta F_j(t)$, $\llang v_i(t_1) \delta F_j(t_2) \rrang = 0$. But we assume that $\llang v_x(0) v_\tta(0) \rrang = \llang v_\tta(0) v_x(0) \rrang = k_B T / m_{x\tta}$. And note $m_x \llang v_x^2(0) \rrang / 2 = k_B T / 2$, $m_\tta \llang v_\tta^2(0) \rrang / 2 = k_B T /2$.


$$ \agn{
& \llang v_x(0) v_x(t) \rrang = \llang v_x(0) \lls v_{x0}(t) + v_{x1}(t) \rrs \rrang = \llang v_x(0) v_{x0}(t) \rrang + \llang v_x(0) v_{x1}(t) \rrang \\ % rang 1-0
&= \frac{k_B T}{m_x} e^{-\gamma_{x0}t} \llp 1 - \gamma_{1,xx} t \rrp + \frac{k_B T}{m_{x\tta}} \frac{ \gamma _{1,x\tta} }{\gamma _{x0}-\gamma _{\tta0}} \llp e^{- \gamma _{x0} t} - e^{- \gamma _{\tta0} t} \rrp \\ % rang 1-2
& \llang v_\tta(0) v_\tta(t) \rrang = \llang v_\tta(0) \lls v_{\tta0}(t) + v_{\tta1}(t) \rrs \rrang = \llang v_\tta(0) v_{\tta0}(t) \rrang + \llang v_\tta(0) v_{\tta1}(t) \rrang \\ % rang 2-0
&= \frac{k_B T}{m_\tta} e^{-\gamma_{\tta0} t} \llp 1 - \gamma_{1,\tta\tta} t \rrp + \frac{k_B T}{m_{x\tta}} \frac{ \gamma _{1,\tta x}}{\gamma _{x0}-\gamma _{\tta0}} \llp e^{- \gamma _{x0} t} - e^{- \gamma _{\tta0} t} \rrp
}$$


Define MSD as $\llang \Delta r_i^2(t) \rrang = \llang \int_0^t \mrm{d}\tau_1 \int_0^t \mrm{d}\tau_2 v_i(\tau_1) v_i(\tau_2) \rrang$. We compute this value by its derivative as a function of $\lla v_i(0) v_i(t) \rra$, since
$$ \dv[]{}{t} \llang \Delta r_i^2(t) \rrang = 2 \int_0^t \mrm{d}\tau \llang v_i(0) v_i(\tau) \rrang $$
%We have the derivatives
%$$ \agn{
%& \dv[]{}{t} \llang \Delta r_x^2(t) \rrang = 2 \int_0^t \mrm{d}\tau \llang v_x(0) v_x(\tau) \rrang = 2 k_B T \times \\ % rang 1
%& \llp \frac{1-e^{-t \gma_{x0}}}{m_x \gma_{x0}} - \frac{\gma_{1,xx} \left(1-e^{-t \gma_{x0}} \left(t \gma_{x0}+1\right)\right)}{m_x \gma_{x0}^2} + \frac{\gamma_{0,x\tta1}}{m_{x\tta} (\gamma_{0,xx0} - \gma_{\tta0})} \llp \frac{1-e^{-\gamma_{0,xx0} t}}{\gamma_{0,xx0}} - \frac{1 - e^{-\gma_{\tta0} t}}{\gma_{\tta0}} \rrp \rrp \\ % rang 2
%} $$
%$$ \agn{
%& \dv[]{}{t} \llang \Delta r_\tta^2(t) \rrang = 2 \int_0^t \mrm{d}\tau \llang v_\tta(0) v_\tta(\tau) \rrang = 2 k_B T \times \\
%& \llp \frac{1-e^{-t \gamma _{0,\tta\tta0}}}{m_\tta \gamma _{0,\tta\tta0}} - \frac{\gamma _{0,\tta\tta1} \left(1-e^{-t \gamma _{0,\tta\tta0}} \left(t \gamma _{0,\tta\tta0}+1\right)\right)}{m_x \gamma _{0,\tta\tta0}^2} + \frac{\gamma_{0,\tta x1}}{m_{x\tta} (\gamma_{0,xx0} - \gma_{\tta0})} \llp \frac{1-e^{-\gamma_{0,xx0} t}}{\gamma_{0,xx0}} - \frac{1 - e^{-\gma_{\tta0} t}}{\gma_{\tta0}} \rrp \rrp
%} $$
After two integrations, we have $\llang \Delta r_i^2 (t) \rrang$

$$ \agn{
\llang \Delta r_x^2(t) \rrang &= \llang \Delta r_x^2(0) \rrang + k_B T \times \left(\frac{\frac{e^{-t \gma_{x0}}-1}{\gma_{x0}}+t}{m_x \gma_{x0}} + \frac{\gma_{1,x\tta} \left(\frac{e^{-t \gma_{x0}}-1}{\gma_{x0}}+t\right)}{\gma_{x0} m_{x\tta} \left(\gma_{x0}-\gma_{\tta0}\right)} - \frac{\gma_{1,x\tta} \left(\frac{e^{-t \gma_{\tta0}}-1}{\gma_{\tta0}}+t\right)}{\gma_{\tta0} m_{x\tta} \left(\gma_{x0}-\gma_{\tta0}\right)} - \frac{\gma_{1,xx} \left(t-\frac{2-e^{-t \gma_{x0}} \left(t \gma_{x0}+2\right)}{\gma_{x0}}\right)}{m_x \gma_{x0}^2}\right) \\ % rang 2
&\overset{t\to0}{\approx} \llang \Delta r_x^2(0) \rrang + \frac{k_B T}{m_x} t^2 - \frac{\left(k_B T \left(m_x \gma_{1,x\tta}+m_{\text{x$\theta $}} \left(\gma_{x0}+\gma_{1,xx}\right)\right)\right)}{3 \left(m_x m_{\text{x$\theta $}}\right)} t^3 % rang 3
} $$

%%%
$$ \agn{
\llang \Delta r_\tta^2(t) \rrang &= \llang \Delta r_\tta^2(0) \rrang + k_B T \times \llp \frac{\frac{e^{-t\gma_{\tta0}} - 1}{\gma_{\tta0}} + t}{m_\tta \gma_{\tta0}} + \frac{\gma_{1,\tta x} \left(\frac{e^{-t \gma_{x0}}-1}{\gma_{x0}}+t\right)}{\gma_{x0} m_{x\tta} \left(\gma_{x0}-\gma_{\tta0}\right)} - \frac{\gma_{1,\tta x} \left(\frac{e^{-t \gma_{\tta0}}-1}{\gma_{\tta0}}+t\right)}{\gma_{\tta0} m_{x\tta} \left(\gma_{x0}-\gma_{\tta0}\right)} - \frac{\gma_{1,\tta\tta} \llp t - \frac{2 - e^{-t \gma_{\tta0}} (t \gma_{\tta0} +2)}{\gma_{\tta0}} \rrp}{m_\tta \gma_{\tta0}^2} \rrp \\ % rang 2
&\overset{t\to0}{\approx} \llang \Delta r_\tta^2(0) \rrang + \frac{k_B T}{m_{\theta }} t^2  - \frac{ \left(k_B T \left(m_{\theta } \gma_{1,\tta x}+m_{\text{x$\theta $}} \left(\gma_{\tta0}+\gma_{1,\tta\tta}\right)\right)\right)}{3 \left(m_{\theta } m_{\text{x$\theta $}}\right)} t^3 % rang 3
} $$


Additionally, cross mean "square" displacement could also been derived between $x$ and $\tta$.

$$ \lla \Delta r_x(t) \cdot \Delta r_\tta(t) \rra = \int_0^t \lls \dv[]{}{t} \lla \Delta r_x(\tau) \cdot \Delta r_\tta(\tau) \rra \rrs \mrm{d}\tau + \lla \Delta r_x(0) \cdot \Delta r_\tta(0) \rra $$

Since $\Delta r_x(t) = \int_0^t v_x(\tau) \mrm{d}\tau$, $\Delta r_\tta(t) = \int_0^t v_\tta(\tau) \mrm{d}\tau$, we have
$$ \dv[]{}{t} \lla \Delta r_x(t) \cdot \Delta r_\tta(t) \rra = \int_0^t \lla v_x(t) v_\tta(\tau) \rra \mrm{d}\tau + \int_0^t \lla v_x(\tau) v_\tta(t) \rra \mrm{d}\tau  $$
Consider cross velocity product average up to 1-order of $\kpa$:
$$ \agn{ & \lla v_x(\tau_1) v_\tta(\tau_2) \rra \approx \\ & \fives \lla v_{x0}(\tau_1) v_{\tta0}(\tau_2) \rra + \lla v_{x0}(\tau_1) v_{\tta1}(\tau_2) \rra + \lla v_{x1}(\tau_1) v_{\tta0}(\tau_2) \rra }$$
Only taking $\lla v_x^2(0) \rra$, $\lla v_\tta^2(0) \rra$, $\lla v_x(0) v_\tta(0) \rra = \frac{k_B T}{m_{x\tta}}$ mentioned previously into account, we insist that $\lla \dlt F_x (\tau_1) \dlt F_\tta (\tau_2) \rra = 0$, and $\lla v_i (\tau_1) \dlt F_j (\tau_2) \rra = 0$. Therefore, we could easily calculate each term:

$$ \agn{ &\lla v_{x0} (\tau_1) v_{\tta0} (\tau_2) \rra = \lla v_x(0) v_\tta(0) \rra e^{-\gma_{0x} \tau_1} e^{-\gma_{0\tta} \tau_2} \\
& \lla v_{x0} (\tau_1) v_{\tta1} (\tau_2) \rra = \lla v_x^2(0) \rra \frac{e^{-\gma_{0x} \tau_1} \gma_{1,\tta x}}{\gma_{0x} - \gma_{0\tta}} \llp e^{-\gma_{0x} \tau_2} - e^{-\gma_{0\tta} \tau_2} \rrp - \lla v_x(0) v_\tta(0) \rra \gma_{1,\tta\tta} \tau_2 e^{-\gma_{0x} \tau_1} e^{-\gma_{0\tta} \tau_2} \\
& \lla v_{x1}(\tau_1) v_{\tta0}(\tau_2) \rra = \lla v_\tta^2(0) \rra \frac{e^{-\gma_{0\tta} \tau_2} \gma_{1,x\tta}}{\gma_{0x} - \gma_{0\tta}} \llp e^{-\gma_{0x} \tau_1} - e^{-\gma_{0\tta} \tau_1} \rrp - \lla v_x(0) v_\tta(0) \rra \gma_{1,xx} \tau_1 e^{-\gma_{0x} \tau_1} e^{-\gma_{0\tta} \tau_2} }$$

We jump the explicit calculation process, giving the final result directly:

$$ \agn{ 
& \lla \Delta r_x(t) \cdot \Delta r_\tta(t) \rra = \lla \Delta r_x(0) \cdot \Delta r_\tta(0) \rra \\ % rang 0
&  + \frac{\gma_{1,\tta x} \left(e^{t \gma_{x0}}-1\right) e^{-t \left(\gma_{\tta0}+2 \gma_{x0}\right)} \left(\gma_{\tta0} e^{t \gma_{\tta0}} \left(e^{t \gma_{x0}}-1\right)-\gma_{x0} \left(e^{t \gma_{\tta0}}-1\right) e^{t \gma_{x0}}\right)}{\gma_{\tta0} \gma_{x0}^2 \left(\gma_{x0}-\gma_{\tta0}\right)} \lla v_x^2(0) \rra \\ % rang 1-1
&  + \frac{\gma_{1,x\tta} \left(e^{t \gma_{\tta0}}-1\right) e^{-t \left(2 \gma_{\tta0}+\gma_{x0}\right)} \left(\gma_{\tta0} e^{t \gma_{\tta0}} \left(e^{t \gma_{x0}}-1\right)-\gma_{x0} \left(e^{t \gma_{\tta0}}-1\right) e^{t \gma_{x0}}\right)}{\gma_{\tta0}^2 \gma_{x0} \left(\gma_{x0}-\gma_{\tta0}\right)} \lla v_\tta^2(0) \rra \\ % rang 1-2
&  + \frac{e^{-t \left(\gma_{\tta0}+\gma_{x0}\right)}}{\gma_{\tta0}^2 \gma_{x0}^2} \lla v_x(0) v_\tta(0) \rra \times \\ % rang 1-3-1
& \fives \left[ \gma_{x0} \left(\gma_{\tta0} \left(\left(e^{t \gma_{x0}}-1\right) \left(e^{t \gma_{\tta0}}+t \gma_{1,\tta\tta}-1\right)+t \gma_{1,xx} \left(e^{t \gma_{\tta0}}-1\right)\right) \right. \right. \\ % rang 1-3-2
& \fives \left. \left. - \gma_{1,\tta\tta} \left(e^{t \gma_{\tta0}}-1\right) \left(e^{t \gma_{x0}}-1\right)\right)-\gma_{\tta0} \gma_{1,xx} \left(e^{t \gma_{\tta0}}-1\right) \left(e^{t \gma_{x0}}-1\right) \right] % rang 1-3-3
}$$





%\ssc*{Mean first passage time}
%Another way of describing diffusion-to-target rates is in terms of first passage times. The mean first passage time (MFPT), ⟨τ⟩, is the average time it takes for a diffusing particle to reach a target position for the first time. The inverse of ⟨τ⟩ gives the rate of the corresponding diffusion-limited reaction. A first passage time approach is particularly relevant to problems in which a description the time-dependent averages hide intrinsically important behavior of outliers and rare events, particularly in the analysis of single molecule kinetics.
%------------------------------------------------


%------------------------------------------------

\section*{Numerical Simulations} %% 2-3 pages



\ssc*{Discretisation algorithm}
\label{Discretisation algorithm}

%If we suppose that $N_{\mrm{max}}$ is the maximum number for simulation steps, so we divide the continuous time $t \in [ 0 , t_{\mrm{max}} ]$ as $t_i \in \{ t_0 = 0, t_1, \cdots, t_{N} = t_{\mrm{max}} \}$ with $i \in [ 1 , N_{\mrm{max}} ]$, where the time gap $\Dlt t = t_{\mrm{max}} / N_{\mrm{max}}$. %$t_r$ the ratio . .
We set $N_{\mrm{max}} = 60000$ for the following numerical simulations. Besides, we introduce another parameter, the time scaling ratio $t_r = 1/200$, which means that we would split the time unit into 200 intervals, for the sake of much smooth numerical results. Consider the real time unit less than the typical time of Brownian motion, for example 1ms; so we use $dt = \frac{1\mrm{ms} \cdot c}{r \sqrt{2\eps}} \cdot t_r$ in the simulation codes.



\sss*{Dimensionless variables}
In order to non-dimensionalize the problem, we follow the variables used in \cite{JFM2015}:
$$ \dlt = \Dlt \cdot r \eps \fives x_G = X_G \cdot r \sqrt{2\eps} \fives \tta = \Theta \cdot \sqrt{2\eps} $$
Introduce $t = T \cdot r \sqrt{2\eps} / c$, where $c$ is the maximum velocity for free fall particles. So the velocities in reality would be replaced by those shown in our equations of motion:
$$ v_\Dlt = \frac{v_z}{c} \cdot \sqrt{\frac{2}{\eps}} \tens v_X = \frac{v_x}{c} \tens v_\Theta = \frac{v_\tta r}{c} $$
where $c$ is a free fall velocity scale constant
$$ c = \sqrt{2 g r \rho^\ast / \rho} $$
and $\rho^\ast = \rho_{\mrm{sty}} - \rho_{\mrm{sol}}$, $\rho = \rho_{\mrm{sol}} = 1.00$ g/cm$^3$, $\rho_{\mrm{sty}} = 1.06$ g/cm$^3$.
Similarly, related to velocities, accelerations and forces would be treated in the same way according to their direction.
$$ \dot{v}_\Dlt = \dot{v}_z \cdot \frac{2r}{c^2} \tens \dot{v}_X = \dot{v}_x \cdot \frac{r \sqrt{2\eps}}{c^2} \tens \dot{v}_\Theta = \dot{v}_\tta \cdot \frac{r^2 \sqrt{2\eps}}{c^2} $$

%$$ F_\Dlt = \frac{F_z}{c} \cdot \sqrt{\frac{2}{\eps}} \tens F_X = \frac{F_x}{c} \tens F_\Theta = \frac{F_\tta r}{c} $$


$$ \lla \widehat{\dlt F_{z}}(\tau_1) \cdot \widehat{\dlt F_{z}}(\tau_2) \rra = \lla \dlt F_{z}(\tau_1) \cdot \dlt F_{z}(\tau_2) \rra \times \frac{2}{c^2 \eps} $$

$$ \lla \widehat{\dlt F_{x}}(\tau_1) \cdot \widehat{\dlt F_{x}}(\tau_2) \rra = \lla \dlt F_{x}(\tau_1) \cdot \dlt F_{x}(\tau_2) \rra \times \frac{1}{c^2} $$

$$ \lla \widehat{\dlt F_{\tta}}(\tau_1) \cdot \widehat{\dlt F_{\tta}}(\tau_2) \rra = \lla \dlt F_{\tta}(\tau_1) \cdot \dlt F_{\tta}(\tau_2) \rra \times \frac{r^2}{c^2} $$




\sss*{Euler-Maruyama method}
In Itô calculus, the Euler–Maruyama method is used for the approximate numerical solution of a stochastic differential equation (SDE). 
%It is an extension of the Euler method for ordinary differential equations to stochastic differential equations. 
%Unfortunately, the same generalization cannot be done for any arbitrary deterministic method. 
%[Kloeden, P.E. & Platen, E. (1992). Numerical Solution of Stochastic Differential Equations. Springer, Berlin. ISBN 3-540-54062-8.]
Consider the equation 
$$ dX_t = a(X_t,t) dt + b(X_t,t) dW_t $$
with initial condition $X_0 = x_0$, where $W_t$ stands for the Wiener process, and suppose that we wish to solve this SDE on some interval of time $[0, T]$. Then the Euler-Maruyama approximation to the true solution $X$ is the Markov chain $Y$ defined as follows:
\begin{itemize}[noitemsep]
	\item partition the interval $[0,T]$ into $N$ equal subintervals of width $\Dlt t = T/N > 0$:
	$ 0 = \tau_0 < \tau_1 < \cdots < \tau_N = T $
	\item set $Y_0 = x_0$
	\item recursively define $Y_n$ for $0 \leqslant n \leqslant N-1$ by
	$$ Y_{n+1} = Y_n + a(Y_n,\tau_n) \Dlt t + b(Y_n,\tau_n) \Dlt W_n $$
	where the random variables $\Dlt W_n$ are independent and identically distributed normal random variables with expected value zero and variance $\Dlt t$.
\end{itemize}


%The Euler-Maruyama scheme is straightforwardly applied to
%$$ \agn{ U(t+\Dlt t) &= U(t) + \llp - \frac{\xi(R(t))}{m} (U(t) - v(R(t))) + \frac{1}{m} F_{\mrm{ext}}(R(t),t) \rrp \Dlt t \\ &\fives + \sqrt{\frac{2\xi(R(t)) k_B T}{m^2}} \Dlt W(\Dlt t) \\
% R(t+\Dlt t) &= R(t) + U(t) \Dlt t } $$


\sss*{Discrete-Time Langevin Integration}
%Based on the article about Discrete-Time Langevin Integration. For multiple dimensions, see its Support Information:
%https://pubs.acs.org/doi/suppl/10.1021/jp411770f/suppl\_file/jp411770f\_si\_001.pdf

Consider a Langevin equation with the external force $f(t)$, namely the gravity and the spurious force for us:
$$ dv = \frac{f(t)}{m} dt - \gamma v dt + \sqrt{\frac{2\gamma}{\beta m}} dW(t) $$
we exploit the discretisation algorithm based on \cite{JPCB2014} with the following splitting steps:
%For this operator splitting, a single update step that advances the simulation clock by $\Delta t$ is given explicitly by
$$ \agn{
& \bbf{v}\left( n+ \frac{1}{4}\right) = \sqrt{a} \cdot \bbf{v}(n) + \left[ \frac{1}{\beta} (\bbf{1} - \bbf{a}) \cdot \bbf{m}^{-1} \right]^{1/2} \cdot \bbf{N}^{+} (n) \\ % rang 1
& \fives \bbf{v} \left( n+ \frac{1}{2}\right) = \bbf{v} \left( n+ \frac{1}{4}\right) + \frac{\Delta t}{2} \bbf{b} \cdot \bbf{m}^{-1} \cdot \bbf{f}(n) \\ % rang 2
& \fives \fives \bbf{r} \left( n+ \frac{1}{2}\right) = \bbf{r}(n) + \frac{\Delta t}{2} \bbf{b} \cdot \bbf{v} \left( n+ \frac{1}{2}\right) \\ % rang 3
& \fives \fives \fives \scr{H}(n) \to \scr{H}(n+1) \\ % rang 4
& \fives \fives \bbf{r} \left( n+1\right) = \bbf{r} \left( n+ \frac{1}{2}\right) + \frac{\Delta t}{2} \bbf{b} \cdot \bbf{v} \left( n+ \frac{1}{2}\right) \\ % rang 5
& \fives \bbf{v}\left( n+ \frac{3}{4}\right) = \bbf{v}\left( n+ \frac{1}{2}\right) + \frac{\Delta t}{2} \bbf{b} \cdot \bbf{m}^{-1} \cdot \bbf{f}(n+1) \\ % rang 6
& \bbf{v}\left( n+1 \right) = \sqrt{a} \cdot \bbf{v}\left( n+ \frac{3}{4}\right) + \left[ \frac{1}{\beta} (\bbf{1} - \bbf{a}) \cdot \bbf{m}^{-1} \right]^{1/2} \cdot \bbf{N}^{-} (n+1) % rang 7
} $$
where $a_{ij} = \delta_{ij} \exp(-\gamma_i \Delta t)$, $\scr{N}^\pm$ are independent normally distributed random variables with zero mean and unit variance, $b_{ij} = \delta_{ij} \sqrt{\frac{2}{\gamma_i \Delta t} \tanh \frac{\gamma_i \Delta t}{2}}$




%\ssc*{Numerical Results}
%Here we would add figures for our numerical results, comparing with the analytical functions expected.







%overdamped diffusion coefficient...

%\sss*{Typical height}
%Here we consider the typical height, in which the diffusion coefficient is just equal to the average value.
%Therefore, we would like to seek a characteristic height $z^\ast$ in this case. Consider the diffusion coefficient $D_z$ as a function of $z,t$. 

%$$ D_z(\Dlt) = \int_0^\infty \lla v_z(0) v_z(t) \rra \llp \Dlt \rrp \mrm{d}t = \int_0^\infty \lla v_z^2(0) \rra \exp \llp - \frac{\xi}{\Dlt^{3/2}} t \rrp \cdot \llp 1 - \frac{15 \kpa \xi^2}{8 \Dlt^4} t \rrp \mrm{d}t $$
%$$ \agn{ \lla D_z \rra &= \frac{1}{z_+ - z_-} \int_{z_-}^{z_+} P(\Dlt) D_z(\Dlt) \mrm{d}\Dlt \\ &= \frac{1}{z_+ - z_-} \int_{z_-}^{z_+} \mrm{d}\Dlt \int_0^\infty \mrm{d}t  \lla v_z^2(0) \rra \exp \llp - \frac{\xi}{\Dlt^{3/2}} t \rrp \cdot \llp 1 - \frac{15 \kpa \xi^2}{8 \Dlt^4} t \rrp } $$

%where $z_-, z_+$ refer to the minimum and maximum height. This average diffusion coefficient would just be corresponding to the value at a given position, namely the typical height $z^\ast$. 
%$$ \lla D_z \rra = D_z(z^\ast) $$





%------------------------------------------------

%\vspace{2cm}
%{\Large \textcolor{red}
%{Contents: at most, 10 pages or 6000 words.}}


%----------------------------------------------------------------------------------------
%	REFERENCE LIST
%----------------------------------------------------------------------------------------
\begin{thebibliography}{99} %% Ten papers listed?

\bibitem{JFM2015}
T. Salez, L. Mahadevan, {\it J. Fluid Mech.} {\bf 2015}, {\it 779}, 181-196, {\bf Elastohydrodynamics of a sliding, spinning and sedimenting cylinder near a soft wall}.

\bibitem{JPCB2014}
D. A. Sivak, J. D. Chodera, and G. E. Crooks, {\it J. Phys. Chem. B} {\bf 2014}, {\it 188(24)}, 6466-6474, {\bf Time step rescaling recovers continuous-time dynamical properties for discrete-time Langevin integration of nonequilibrium systems}.


\end{thebibliography}
%----------------------------------------------------------------------------------------


\end{document}
